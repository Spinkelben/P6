\section[Udviklingsmetode]{Udviklingsmetode}

% Emneoverskrift. Start jeres del med denne:
\begin{frame}
  \frametitle{}
  \begin{center}
    {\Huge Udviklingsmetode}
  \end{center}
\end{frame}
\note{
  \begin{itemize}
		\item Notes...
  \end{itemize}
}

% Indhold:
\begin{frame}
    \frametitle{Udviklingsmetode i vores gruppe}
    \includegraphics[width=0.48\textwidth]{burndown}~~~~
    \includegraphics[width=0.48\textwidth]{kanban}
    
\end{frame}
\note{
	\begin{itemize}
		\item Scrum
    \item Det passer godt ind i multiprojektet.
    \item Daily Scrum
    \begin{itemize}
      \item Vælger opgaver fra sprint backloggen
      \item Opdaterer burndown
    \end{itemize}
    \item Scrumboardet (kanban): valgte backlog items > opsplitning i opgaver > planning poker. Enhed: halvdage. 
    \item Derudover følger vi udvkl.metode i multiprojektet.
    \item TODO: Hvad gør vi, når vi, som på billedet, er bagefter?
	\end{itemize}
}

\begin{frame}
    \frametitle{Udviklingsmetode i multiprojektet}
    \centering
    \includegraphics[width=0.8\textwidth]{multiproject_illustration.pdf}
\end{frame}
\note{
	\begin{itemize}
    \item 15 grupper, 3 subprojekter.
    \item En agil metode er et godt valg: ingen udviklererfaring med kode+krav+kunder, og vi har ikke tid til at analysere og forstå den nuværende kodebase. Vi skal bruge tiden på at få den til at virke!
    \item Figur:
    \begin{itemize}
      \item Multiprojekt-niveau: Sikre sig at roller, arbejdet og møderne overholdes. 1 ugentligt møde
      \item Subprojekt: Sprint planning, sprint review. Derudover 2-3 møder pr. uge.
      \item Gruppe: Scrum of (Scrums-something). Grupperne skal implementere interfacet Scrum. Hvad de gør, er vi ligeglade med.
    \end{itemize}
	\end{itemize}
}

\begin{frame}
    \frametitle{Udviklingsmetode i multiprojektet}
    \centering
    TODO: grafik med arbejdsprodukter.
\end{frame}
\note{
	\begin{itemize}
    \item Product backlog: Kundernes krav og ønsker.
    \item Release backlog: Nogle af disse ønsker flyttes over i en release backlog, og der arbejdes på dem i nuværende sprint.
	\end{itemize}
}

\begin{frame}
    \frametitle{Udviklingsmetode i multiprojektet}
    \centering
    TODO: grafik med backlog items.
\end{frame}
\note{
	\begin{itemize}
    \item Feature
    \item Bug
    \item Constraint
    \item Knowledge Acquisition
    \item Technical Work
    \item As a <type of user>, I want <some goal> so that <some reason>.
	\end{itemize}
}

%%%%%%%%%%%%%%%%%%%%%%%%%
%% SCM
%%%%%%%%%%%%%%%%%%%%%%%%%
\begin{frame}
  \frametitle{}
  \begin{center}
    {\Huge Konfigurationsstyringsplan}
  \end{center}
\end{frame}
\note{
  \begin{itemize}
		\item Notes...
  \end{itemize}
}