%!TEX root = ../report.tex
\chapter{Software Configuration Management Plan}\label{chap:sw_intro_cm}
Software configuration management is an important part of any large software project. Configuration management can be defined as:

\begin{itemize}
  \item []\emph{$\ldots{}$ the discipline of identifying the configuration of a system at distinct points in time for the purpose of systematically controlling changes to the configuration and maintaining the integrity and traceability of the configuration throughout the system life cycle} \parencite[ch.6, p.6-1]{swebok}
\end{itemize}

The large size of the multi-project requires some form of software configuration management to handle and track the changes in the software. It can ease the workflow of developers and assure the quality of the product \parencite[ch.6]{swebok}.

Our role as a group is to manage the development method, and therefore we have the responsibility of creating a Software Configuration Management Plan (SCMP). An SCMP specifies how items are controlled, by who, and by what tools \parencite[ch.6]{swebok}.

This chapter describes how we plan and execute configuration management. We aim to make development easy for the developers, and therefore aim to automate as much as possible.

\begin{chapterorganization}
  \item in \sectionref{sec:SCM_orgcontext} we describe the organizational context of the multi-project;
  \item in \sectionref{sec:SCM_configitems} we identify which items are under configuration management and describe how they are managed;
  \item in \sectionref{sec:SCM_ci} we describe the continuous integration practices we use;
  \item in \sectionref{sec:SCM_tools} we list the specific tools we use for configuration management.
\end{chapterorganization}

\section{Organizational Context}\label{sec:SCM_orgcontext}
In order to plan the software configuration management it is important to understand the organizational structure \parencite[ch.6]{swebok}. The multi-project has a quite unique structure. The development is performed and managed by students who work in small groups that each acts as an independent entity. While the product is being developed for the clients, the groups have other priorities than customer satisfaction. Since the work is being done for free, the motivation of the groups is to work on something interesting such that they can write a good report detailing the work and get a good grade. While we can suggest a process with specific procedures, we have no authority over the other groups. This means that the  preferences of the groups weigh heavily when we select which software configuration management procedures to follow. In general, the consensus of the multi-project groups is that the less formal communication and bureaucracy the better. This is reflected in the choice of Scrum as the basis for the project management, and will influence all of the decision regarding the software configuration management process planning and execution.

\section{Configuration Items}\label{sec:SCM_configitems}
A configuration item is a piece of software or a combination of hardware and software which is managed as a single entity \parencite[ch.6]{swebok}. The identified items and how they are managed are:

\begin{description}
  \item[Requirements] The requirements are managed by the subproject product owners as described in \sectionref{sec:project_overview}.
  \item[Released Applications] The currently deployed version of all applications is controlled by the group responsible for Google Play. The current version of internally built applications is controlled by us through automatic building. Changes made to an application trigger building and testing. We ensure that the tests are run automatically, but it is the responsibility of the developers of each application to write the tests. The iOS applications developed during previous years are not under configuration management, as it was decided at the initial meeting that those applications will not be worked on.
  \item[Released Libraries] The binaries of all libraries are stored in a Maven repository. When a library builds successfully the binary is uploaded to the repository as a snapshot version. When the developers of the libraries decide that a new release should be made available, new versions of the libraries are uploaded. This way several versions of library binaries are stored.
  \item[The Supporting Tools Gradle and Android Studio] The developers on the multi-project use Android Studio as the main development tool for the Android applications. Android Studio uses Gradle as its build automation system. The versions of Android Studio and Gradle are maintained by the \bd groups and any decision to upgrade is theirs. When the project started Android Studio and Gradle was upgraded. The used version of Android Studio is 1.0.2 and 2.2.1 for Gradle.
  \item[Application and Library Source Code] The source code for the applications and libraries is controlled with Git \parencite{gitwebsite}. Each application and library has its own repository. The external dependencies are managed with Gradle which gets the correct version of each external library from the Maven repository.
  \item[Source Code Documentation] Documentation of source code is written by the developers of the software in Javadoc style. There is no formal process ensuring that it is actually written, and so it is solely up to the developers to do so. Documentation is automatically generated.
\end{description}

\section{Continuous Integration}\label{sec:SCM_ci}
Continuous integration is used to tie together the items under configuration management. Applications and libraries are automatically tested and built so we always have the most recent versions available for the customer. Developers integrate with the master branch frequently to avoid large, error-prone merge conflicts at the end of a sprint. Documentation is automatically generated daily so it is always up-to-date. Quality metrics of the source code, such as test code coverage and potential problems found using static code analysis, are automatically generated. To make continuous integration work in practice, developers are expected to integrate with the master branch daily, write automatic tests, and test code locally before integrating.

\section{Tools}\label{sec:SCM_tools}
We use the following tools to manage the configuration items identified:

\begin{description}
  \item[Git] Git is used as the version control as mentioned previously. This was set up by previous years, and we continue to use their setup. The Git group manages Git.
  \item[Jenkins] Jenkins is used to automatically build and test applications. Jenkins is detailed in \sectionref{sec:jenkins}. Jenkins is used for all applications throughout the multi-project, except for the iOS applications. We manage Jenkins.
  \item[Artifactory] The Artifactory tool is a Maven repository that stores various versions of binaries of the libraries in the multi-project. Artifactory, managed by us, is described in more detail in \sectionref{sec:dependency_repository}.
  \item[Doxygen] We use Doxygen to generate source code documentation. The use of Doxygen is described further in \sectionref{sec:automated_documentation_gen}. Doxygen is managed by us, but the documentation itself is managed by the developers.
\end{description}