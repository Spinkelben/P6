%!TEX root = ../report.tex
\chapter{Introduction}
\kimnote{Et projekt skal starte med en general forklaring af problemstillingen i den virklige verden. Derefter kan i nævne formelle ting.}
The purpose of this semester project is to develop a complex software system in a large development environment. The software system is called GIRAF (Graphical Interface Resources for Autistic Folk) and is inherited from the 6\textsuperscript{th} semester students of past years. The project is a collaboration between Aalborg University, Aalborg Municipality and several institutions that work with autistic citizens. 
\kimnote{I skal beskrive hvem der præcis er jeres kunder og ikke bare nogle af dem. Det er nok bedst er flytte sådan en redegørelse et andet sted hen.} \todo{Skriv noget om ``external customer'' da vi også har ``internal customers''}

The GIRAF project was initiated by Ulrik Nyman, Associate Professor at Aalborg University, in 2011. It is a software suite aimed at easing the daily routines of autistic citizens and their guardians. Autistic citizens generally have limited language skills and thus their primary way of communicating is through pictograms. An important aspect of the multi-project is therefore easing this communication.

The project consists of several front-end and back-end subprojects, that each serves a purpose of the combined system. Most of the front-end subprojects are Android apps and the back-end subprojects are libraries and database interfaces. Examples of the Android apps are \todo{Insert example applications with brief description}.

%\begin{description}
%  \item[Launcher] \dummy
%  \item[Sekvens]
%  \item[Pictooplæser]
%	\item[Kategoriværktøjet] \dummy
%  \item[Picto Creator] Hvad er dette?
%  \item[Pictotegner]
%  \item[Picto Search] \dummy
%  \item[Oasis App]
%  \item[Ugeplan]
%  \item[Tidstager] 
%  \item[Stemmespillet]
%  \item[Kategorispillet]
%  \item[Web Ugeplan]
%  \item[Webadmin]
%\end{description}
%\todo{This list are the front-end apps. Maybe move to appendix?}
%\kimnote{Forklar det der er essentielt for jeres projekt. Det kunne være fint med en liste over de andre "projekter" i appendix, men brug kun tid på det hvis det bringer værdi til jeres projekt. Det der skal stå i en rapport er hvad i har lavet krydret en passende mængde context.}

There are 14 project groups with approximately 4 students in each. It is crucial that project groups work towards creating a working system, and to allow this a common work process has been decided by the semester coordinator. As such, the semester is split into four sprints (in Scrum terms).
%We choose to call the first sprint for Sprint 0, as we consider it an initial start-up sprint where we allow ourselves to understand the multi-project. \kimnote{I bør introducere jeres process(scrum) før i begynder at bruge fagtermer fra den. Det er nok med et par linjer her i intruduktionen hvor i skriver at i bruger scrum. Jeg tror det vil gavne jer at have en kort sektion om jeres adoptering af scrum, altså hvilke scrum værktøjer i bruger, medmindre i bruger ren scrum!}
The organization of this report reflects these sprints. \todo{Write more about early process decisions.}

\todo{Skriv at vi arbejder med proces og SW CM hele projektet, og derfor har vi kapitler i starten accordingly.}

\section{Project Dates}
\tikzsetnextfilename{gantt_dates}
\begin{tikzpicture}
	\begin{ganttchart}[x unit=1.14mm, time slot format=isodate]{2015-02-10}{2015-05-28}
    \gantttitlecalendar{month=name}\\
     
    \ganttgroup{Sprint 1}{2015-02-15}{2015-03-10}\\
    \ganttgroup{Sprint 2}{2015-03-12}{2015-04-08}\\
    \ganttgroup{Sprint 3}{2015-04-15}{2015-04-27}\\
    \ganttgroup{Sprint 4}{2015-05-04}{2015-05-20}
    
  \end{ganttchart}
\end{tikzpicture}

\kimnote{I love this figure. Consider placing it together in the section meeting dates and then add sprint end etc. You should also talk about the "holes" between sprint, where the GUI do sprint planning.}
Each sprint has approximately 8 full days per man.


\begin{documentorganization}
  \item in \chapterref{chap:sw_dev_method} we describe the software development method used across the multi-project as well as in our group. The method described is the final method which has been refined through each sprint;
  \item in \chapterref{chap:sw_intro_cm} we investigate what software configuration management means. Elaborating on this, we describe the organizational context and configuration items and tools.
  \item Sprint 1:
  \begin{itemize}
    \item In \chapterref{chap:sprint1_planning} we describe the sprint planning process;
    \item In \chapterref{chap:config_management} we ... ;
    \item In \chapterref{chap:sprint1_end} we conclude on the sprint. 
  \end{itemize}
  \item Sprint 2:
  \begin{itemize}
    \item In \chapterref{chap:s2_sprintplanning} we describe the sprint planning process;
    \item In ... \dummy
    \item In \chapterref{chap:sprint2_end} we conclude on the sprint. 
  \end{itemize}
  \item Sprint 3:
  \begin{itemize}
    \item In \chapterref{chap:s3_sprintplanning} we describe the sprint planning process;
    \item In ... \dummy
    \item In \chapterref{chap:sprint3_end} we conclude on the sprint. 
  \end{itemize}
  \item Sprint 4:
  \begin{itemize}
    \item \todo{Fill Sprint 4 content in.}
  \end{itemize}
\end{documentorganization}