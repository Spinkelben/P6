\chapter{Introduction}

The purpose of this semester project is to develop a complex software system in a large development environment. The software system is called GIRAF (Graphical Interface Resources for Autistic Folk) and is inherited from the 6\textsuperscript{th} semester students of past years. The project is a collaboration between Aalborg University, Aalborg Municipality and several institutions that work with autistic citizens. 

The GIRAF project was initiated by Ulrik Nyman, Associate Professor at Aalborg University, in 2011. It is a software suite for Android tablets aimed at easing the daily routines of autistic citizens and their guardians. The project consists of several front-end and back-end subprojects, that each serves a purpose of the combined system. Examples of the subprojects are \todo{Insert example applications with brief description}.

\begin{description}
  \item[Launcher] \dummy
  \item[Sekvens]
  \item[Pictooplæser]
	\item[Kategoriværktøjet] \dummy
  \item[Picto Creator] Hvad er dette?
  \item[Pictotegner]
  \item[Picto Search] \dummy
  \item[Oasis App]
  \item[Ugeplan]
  \item[Tidstager] 
  \item[Stemmespillet]
  \item[Kategorispillet]
  \item[Web Ugeplan]
  \item[Webadmin]
\end{description}
\todo{This list are the front-end apps. Maybe move to appendix?}

\todo{Write that the semester is divided into 14 groups of approx. 4 people and that these groups have to work together on GIRAF.}

It is crucial that project groups work towards creating a working system, and to allow this a common work process has been decided by the semester coordinator. As such, the semester is split into four sprints (in Scrum terms). The organization of this report reflects these sprints. \todo{Write more about early process decisions.}

\begin{documentorganization}
  \item Sprint 1:
  \begin{itemize}
    \item In \chapterref{chap:sw_dev_method} we describe our responsibility as the process group;
    \item In \chapterref{chap:sprint1_planning} we describe the sprint planning process;
    \item In \chapterref{chap:config_management} we describe our responsibility as the configuration management group;
    \item In \chapterref{chap:sprint1_end} we describe and conclude on the sprint. 
  \end{itemize}
  \item Sprint 2:
  \begin{itemize}
    \item \todo{Fill Sprint 2 content in.}
  \end{itemize}
  \item Sprint 3:
  \begin{itemize}
    \item \todo{Fill Sprint 3 content in.}
  \end{itemize}
  \item Sprint 4:
  \begin{itemize}
    \item \todo{Fill Sprint 4 content in.}
  \end{itemize}
\end{documentorganization}

\section{Group Role in Multi-Project}
% Fra ulrik:
Ulrik recommends: A chapter describing the role of this project in the setting of the multi-project. This includes ``an analysis of an organizational context''