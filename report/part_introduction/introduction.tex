%!TEX root = ../report.tex
\chapter{Introduction}
Every year, children and young adults are diagnosed with Autism Spectrum Disorders (ASD). There are no known cures, and treatment focuses on improving independence and quality of life \cite{Myers01112007}. This project is a collaboration with several institutions, located in Aalborg Municipality, which offer care to children or adults who suffer from ASDs. The goal of the project is to ease the workload of the caregivers at these institutions and help persons with ASDs become more independent. Currently the caregivers have a selection of tools and methods which they use in the treatment of the ASDs. Some of these lend themselves to digitalization in the form of tablet applications, apps. The role of the caregivers in this collaboration is to select which tools and methods to digitize and to help us do it by providing knowledge of the domain. They will also be the ones who eventually will be using the resulting system, so they act as the customers in this collaboration. Our role in this collaboration is to develop these apps. The system is called GIRAF (Graphical Interface Resources for Autistic Folk) and is inherited from the 6\textsuperscript{th} semester students of past years. Every 6\textsuperscript{th} semester software student of 2015 are working on the GIRAF project.

The GIRAF project was initiated by Ulrik Nyman, Associate Professor at Aalborg University, in 2011. The system provides number of functionalities, but the main focus is assisting with planning and communication. Citizens with ASDs often have limited language skills and learn to enhance their communication with pictograms. There are fifteen project groups which participate in the multi-project: Eleven groups of four students, one group of three, one group of two and two groups with only a single member each. It is crucial that project groups work together towards creating a working system, and to allow this the semester coordinator has decided that the project should be agile and have four iterations. The organization of this report reflects this.

\section{Bootstrapping the project}
While we do start the project with an existing code base, we need to define how we should structure the work. We have an initial meeting where the apps are presented, and the status of the project is explained. We then start a discussion of how to organize the groups. We decide to use the SCRUM process for defining how work is prioritized and assigned. The process have evolved during the four iterations and the next chapter will describe the process as it was by the end of the last iteration.

\todo{Maybe more suited to the reading guide?}
\begin{documentorganization}
    \item in \chapterref{chap:sw_dev_method} we describe the software development method used across the multi-project as well as in our group. The method described is the final method which has been refined through each sprint;
    \item in \chapterref{chap:sw_intro_cm} we investigate what software configuration management means. Elaborating on this, we describe the organizational context and configuration items and tools.
    \item Sprint 1 
    \item Sprint 2
    \item Sprint 3
    \item Sprint 4
    \item Evaluation of the project and recommendations for future developers. \todo{Add actual chapter references}
\end{documentorganization}

\tikzsetnextfilename{gantt_dates}
\begin{tikzpicture}
	\begin{ganttchart}[x unit=1.14mm, time slot format=isodate]{2015-02-10}{2015-05-28}
    \gantttitlecalendar{month=name}\\
     
    \ganttgroup{Sprint 1}{2015-02-15}{2015-03-10}\\
    \ganttgroup{Sprint 2}{2015-03-12}{2015-04-08}\\
    \ganttgroup{Sprint 3}{2015-04-15}{2015-04-27}\\
    \ganttgroup{Sprint 4}{2015-05-04}{2015-05-20}
    
  \end{ganttchart}
\end{tikzpicture}

\todo{Find the good spot for this figure}
\kimnote{I love this figure. Consider placing it together in the section meeting dates and then add sprint end etc. You should also talk about the "holes" between sprint, where the GUI do sprint planning.}
Each sprint has approximately 8 full days per man.


