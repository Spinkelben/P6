%!TEX root = ../report.tex
\chapter{Introduction}
Every year, children and young adults are diagnosed with Autism Spectrum Disorders (ASD) \cite{autism_diagnosis_adults,autism_diagnosis_children}. There are no known cures, and treatment focuses on improving independence and quality of life \cite{Myers01112007}. This project is a collaboration between the Software students on \nth{6} semester at Aalborg University and several institutions, located in Aalborg Municipality, which offer care to children or adults who suffer from ASDs. The goal of the project is to ease the workload of the caregivers at these institutions and help persons with ASDs become more independent. Currently, the caregivers have a selection of tools and methods which they use in the treatment of the ASDs. Some of these lend themselves to digitalization in the form of tablet applications (\emph{apps}). The role of the caregivers in this collaboration is to select which tools and methods to digitize and to help us do it by providing knowledge of the domain. They will also be the primary users of the resulting system, so they act as the \emph{external customers} in this collaboration. The role of the Software students in this collaboration is to understand their requirements and needs and develop the apps.

The overall system is called GIRAF (Graphical Interface Resources for Autistic Folk) and was initiated by Ulrik Nyman, Associate Professor at Aalborg University, in 2011. Since then, the \nth{6} semester students every year have been working on the project. The main purpose of the system, which is nonprofit and open source, is assisting with planning and communication. Citizens with ASDs often have limited language skills and learn to enhance their communication with pictograms. This year, 51 students work on the project, distributed among fifteen project groups: Eleven groups of four students, one group of three, one group of two, and two groups with only a single member each. It is crucial that project groups work together towards creating a working system. This serves as a great challenge for us, as each group has previously worked independently on their own projects. In addition, we take over an existing code base, which we are to understand and develop before we hand it over to the students next year. This requires that code is well documented and easy to understand, which stands in contrast to the code we have developed in previous projects. The tacit knowledge we acquire ourselves will not be passed to the next year's students unless we document it.

\section{Bootstrapping the Project}
While we do start the project with an existing code base, we need to define how we structure the work. We have an initial meeting where the apps are presented and the status of the project is explained.

This is followed by a discussion of how to organize the groups. The semester coordinator recommends that the project should be agile and have four iterations. We follow this recommendation as he has previous experience with the project and knows what has previously worked well. Because most groups have experience with the agile development method \emph{Scrum}, and everyone has been introduced to it in the Software Engineering course, we initially decide to follow this method, which defines how work is prioritized and assigned. In addition, the semester coordinator explains that Scrum has worked well for the GIRAF project the previous years. The specific development method has evolved during the four iterations and the next chapter will describe the method as it was by the end of the last iteration.

In this meeting, we also distribute different roles of the group. These are described in more detail in \chapterref{chap:sw_dev_method}. Our group undertake the responsibility for Jenkins \cite{JenkinsWebsite}, which is the tool used for continuous integration.

\section{Project Goals}
In an initial meeting with one of the external customers, we identify that the most important desire for the GIRAF project is that it the existing apps become complete. The external customer sees no need for new apps, but instead that bugs get fixed in the current apps so that they become stable. In addition, the customer wants the apps to be similar in the user interface such that it feels more like a suite of related apps. It is the overall goal of the project to accommodate these desires from the customer.

The goal of our group is to improve build automation and the testing facilities of the project. This can help discover bugs in the code, which is important regarding the overall goal of the project.

\section{Report Organization}
In \chapterref{chap:sw_dev_method}, we describe the software development method used across the multi-project as well as in our group. The method described is the final method which has been refined through each sprint. This is followed by \chapterref{chap:sw_intro_cm}, in which we present the Software Configuration Management Plan used in the project. Elaborating on this, we describe the organizational context and the items and tools under configuration management.

The next parts of the report describes the four sprints. This includes descriptions of the user stories we solve and the technical work we make. In sprint 1, we describe the reasoning behind the initial development method and the setup of continuous integration. In sprint 2, we improve the build automation further by refactoring how internal libraries are referenced in projects. In the next sprint, sprint 3, we analyze and evaluate the development method. In sprint 4, we improve the test environment by testing on physical devices instead of emulators  and prepare the handover of the project to next year's students.

\chapterref{chap:evaluation} evaluates what we have experienced during this project and finally, in \chapterref{chap:future_dev_recommendations}, we present a number of recommendations for future developers of the GIRAF project.