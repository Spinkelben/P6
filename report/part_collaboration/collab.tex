\chapter{Collaboration 1}%\label{chap:...}
At least two topic focused chapters describing common activities performed with at least
one (maximum three) other group (not the same for each chapter). These chapters can be
written together with other groups such that they are identical in the different reports. These
chapters should mainly be focus on the final state of affairs and not the steps along the way.
They should be the ideal starting point for new developers to continue the development on
the system. It is emphasized that a project cannot cover all of the topics listed in the study
regulation list and that this should not be penalized. These chapters should be focused on
relevant problems in the project, such as:
\begin{itemize}
\item Project management
\item Requirements analysis
\item Requirements management
\item Prototyping
\item Databases
\item System architecture, common class diagrams
\item Usability; usability design, usability test
\item Test and verification; integration test, acceptance test, regression testing, protocol
verification
\end{itemize}

\chapter{Collaboration 2}%\label{chap:...}

\begin{figure}%
  \centering
  \tikzsetnextfilename{version_management}
  \begin{tikzpicture}
    \path (0,0)
      node[cloud, cloud puffs=14.2, cloud ignores aspect, minimum width=2.8cm, minimum height=1.3cm, align=center, draw, anchor=west] (git-cloud) at (0, 0) {Git repository}
      -- (\textwidth/2,0) node[cloud, cloud puffs=13.6, cloud ignores aspect, minimum width=2.3cm, minimum height=1.3cm, align=center, draw] (jenkins-cloud) {Jenkins}
      -- (\textwidth-1mm,0) node[cloud, cloud puffs=14.8, cloud ignores aspect, minimum width=2.6cm, minimum height=1.3cm, align=center, draw, anchor=east] (art-cloud) {Artifactory};

    \draw[->, dashed] (git-cloud) -- node[anchor=south] {Trigger build} (jenkins-cloud);
    \draw[->, dashed] (jenkins-cloud) -- node[anchor=south] {Release artifact} (art-cloud);
    
    %\node[draw, minimum height=1cm, minimum width=3cm] (dev machine) at (0, 4) {Dev machine};
  \end{tikzpicture}
  \caption{}%
  \label{}%
\end{figure}

% Chapters
%\input{part_spring1/somefile}