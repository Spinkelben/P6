\chapter{Process Improvements}
In this sprint, we work on specifying the development method for the multi-project, continuous integration and a guide on how to make Android UI tests.

\begin{chapterorganization}
  \item In \dummy;
\end{chapterorganization}

\section{Specification of Development Method}
While we have specified much of the development method in our report, there is currently only a very sparse specification available for the multi-project groups on the Redmine wiki. We therefore publish our specification in \chapterref{chap:sw_dev_method} (excluding \sectionref{sec:swmethod_ourgroup}) for the multi-project to consult.

In connection with specifying the development method, \group{1} approach us with some feedback on the development method. They make an analysis of some aspects of the multi-project where they feel the development method is not appropriate\cite{processanalysis}\todo{Hvordan refererer vi til dette?}. The main points are: \todo{Jeg synes der mangler lidt mere argumentation}
\begin{description}
  \item[Dividing by Responsibility, Not Functionality] The multi-project divides groups into three subprojects, each with a distinct responsibility. The analysis argue that this makes it hard to formulate user stories and display value to customers \cite{processanalysis}. We have previously discussed the organization in \sectionref{sec:s1_processeval}, and decided it was the best possible structure for our conditions. We discuss the issue with \group{1} and come to an agreement that making a precise specification of how user stories are written will be helpful.

We also agree that the subproject structure has some clear weaknesses, in particular as there is a danger of gold-plating because. All groups in a subproject must have work to do, even though that work may not be very important overall in the multi-project. We therefore suggest not to divide subprojects this way in future semesters, but rather use a single backlog from where all groups can select any item. We expect that we, to a higher degree than today, will be able to give the costumer what they need most.
  \item[No Real Product Owner] There seems to be a misunderstanding of the product owners of the multi-project: In \group{1}'s analysis, they mention a top level product owner (being the external customers), yet there is no such role in the multi-project. They also argue that the subproject product owners (POs) do not sufficiently understand the requirements and mention that there should be a single person being the product owner to avoid conflicting prioritization \cite{processanalysis}.

We do, however, not believe that it would be possible for a single person to be PO, as this person would then be fully occupied with this, which is not desirable in an educational context. Having spoken to the product owners we do not agree that they do not understand the items in the backlog. This is likely due to a misunderstanding of the development method that should be remedied by having a precise specification available.
  \item[The Product Is Seeing Very Limited Use] The analysis states that despite the Giraf apps having been in development since 2011, there is no major use of the system \cite{processanalysis}. While this is indeed correct, it is a mistake to think that a sprint in Scrum should result in a releasable product. It might take many sprints before this can be done \parencite{larman2003}. \group{1} suggest defining a \emph{minimum viable product} (MVP) that has exactly the features that allows the product to be deployed. We discuss this suggestion with \group{1} and agree that this will help us focus on the most important parts of the project. We add the definition of such a product to the backlog, so that it may be worked on in the next sprint.
  \item[Only Features and Bugs in the Product Backlog] The analysis mentions that user stories used in the backlog only described features and bugs, making it difficult to formulate, prioritize, and work on technical work, such as major refactoring, and knowledge acquisition, such as investigating what library to use to solve a certain problem \cite{processanalysis}. We discuss with \group{1} and agree that we add new types of items to the backlog and specify these. A result of having these different types in the backlog is that we make it clear for project members that it is acceptable to spend time on knowledge aquisition and not only technical work.
\end{description}

To summarize, together with \group{1}, we decide to make the following suggestions for next semester:

\begin{description}
  \item[Divide by Functionality] The multi-project should not be structured by subprojects as this semester, but rather every group should be able to pick any item of a common backlog. They should still use the hierachical scrum-of-scrums structure to lessen workload on product owners.
  \item[Work on a Minimum Viable Product] Create an MVP to get the system in use quickly and satisfy the customer and thus get valuable feedback from the users.
\end{description}

We make the following changes to the development method for the final sprint in this semester:

\begin{description}
  \item[A Single Backlog] Currently, the backlog is separately managed by the subproject POs in separate lists. To make it easy to work with the backlog a single backlog should be made available. This is also necessary for the next semester to easily move onto a structure of functionality division.
  \item[New Backlog Items] We add technical work and knowledge acquisition items to the backlog to make it easy to formulate and work on those items.
\end{description}

\section{Specification of Continuous Integration}
We have committed ourselves to solve a user story proposed by the developers, which states that they want a guide about continuous integration. We observe that many developers \emph{want} continuous integration (it was a high priority user story in sprint 1), but do not really understand how to use it. In the sprints up to now, we have observed developers working on features on isolated branches for several weeks without integrating it with the master branch. Likewise, only a single group has written automated tests for their code. This is a big concern, because agile development and continuous integration highly depends on automated testing. In sprint 1, we wrote a sample test for each project to make it easy for developers to start writing tests. In sprint 2, we added a code coverage measure to each project. We find, however, that this has not resulted in more tests written. Therefore, it seems not be enough to simply write how to do. For this continuous integration guide, we try to argue more about why to do as we propose in order to convince developers that they should indeed follow the guidelines.

The CI guide has been released on the Redmine wiki, and is included in \appendixref{app:ci_guide}. Instructions on when to release to Artifactory has been added to the Redmine guide on dependencies. The instructions appears as specified in \sectionref{subsec:release-management}.

\section{Guide on Making Android UI Tests}
To create awareness of the opportunity for groups to have UI tests in the apps, we write a short guide on how to do this, as requested by the multi-project groups. This includes both running UI tests locally as well as in Jenkins upon each build. We do not want to spend much time on doing this, since a great amount of material is already available on the Internet. Therefore, we only find the appropriate material as well as comment on special considerations for our project. These special considerations include how to setup the database with dummy data, and faking login credentials.

The Android UI test guide can be seen in \appendixref{app:uitestguide}.