%!TEX root = ../report.tex
\chapter{Process Improvements}\label{sec:process_improvements}
In this sprint, we work on specifying the development method for the multi-project, continuous integration and a guide on how to make Android UI tests.

\begin{chapterorganization}
  \item In \sectionref{sec:processspecs} we review our development method, and make our specification available to all groups;
  \item In \sectionref{sec:specsCI} we create a guide for the other groups on how to use continuous integration;
  \item In \sectionref{sec:uitestguide} we describe the creation of a guide to performing UI testing.
\end{chapterorganization}

\section{Specification of Development Method}\label{sec:processspecs}
While we have specified much of the development method in our report, there is currently only a very sparse specification available for the multi-project groups on the Redmine wiki. We therefore publish our specification in \chapterref{chap:sw_dev_method} (excluding \sectionref{sec:swmethod_ourgroup}) for the multi-project to consult.

In connection with specifying the development method, \group{1} approaches us with some feedback on the development method. They have made an analysis of some aspects of the multi-project where they feel the development method is not appropriate \cite{processanalysis}\todo{Indsæt som appendix} The main points are: \todo{Jeg synes der mangler lidt mere argumentation i følgende liste:}
\begin{description}
  \item[Dividing by Responsibility, Not Functionality] The multi-project divides groups into three subprojects, each with a distinct responsibility. The analysis argues that this makes it hard to formulate user stories and display value to customers \cite{processanalysis}.

 We agree, that it is hard to formulate user stories. However we do not think this is caused by the multi-project structure, instead we believe it is difficult because it is not clear what a user story is. We discuss the problem with \group{1} and we agree that making a precise specification of how user stories are written will alleviate the difficulty concerning the formulation of user stories.

We have previously discussed the organization of the multi-project in \sectionref{sec:s1_processeval}, and at the time we decided that it was the best possible structure for our conditions. Now, however,  we agree with \group{1} that the subproject structure has some clear weaknesses, in particular as there is a danger of gold-plating. In a subproject, all groups must have work to do, but the workload across subprojects is not the same and changes throughout the project. This means that some groups may work on unimportant things (from the viewpoint of the success of the multi-project). We therefore suggest not to divide subprojects this way in future semesters, but rather use a single backlog from which all groups can select any item. This ensures that work can be selected in a prioritized way. We expect that we, to a higher degree than today, will be able to give the costumer what they need most. We are aware that this may decrease the overall productivity (in terms of functionality over time), as teams do not necessarily specialize in a single technology. However, because teams commit themselves to solve certain user stories (in contrast to being assigned to solve them), we expect that teams generally tend to work on user stories that they feel comfortable with. We accept a small decrease in the overall functionality/time ratio if results in giving the costumers a usable product faster.

  \item[No Real Product Owner] There seems to be a misunderstanding of the product owners of the multi-project: In \group{1}'s analysis, they mention a top level product owner (being the external customers), yet there is no such role in the multi-project. They also argue that the subproject product owners (POs) do not sufficiently understand the requirements and mention that there should be a single person being the product owner to avoid conflicting prioritization \cite{processanalysis}.

We do, however, not believe that it would be possible for a single person to be PO, as this person would then be fully occupied with this, which is not desirable in an educational context. Having spoken to the product owners we do not agree that they do not understand the items in the backlog. This is likely due to a misunderstanding of the development method that should be remedied by having a precise specification available.
  \item[The Product Is Seeing Very Limited Use] The analysis states that despite the Giraf apps having been in development since 2011, there is no major use of the system \cite{processanalysis}.

While this is indeed correct, it is a mistake to think that a sprint in Scrum should result in a releasable product. It might take many sprints before this can be done \parencite{larman2003}. \group{1} suggest defining a \emph{minimum viable product} (MVP) that has exactly the features that allows the product to be deployed. We discuss this suggestion with \group{1} and agree that this will help us focus on the most important parts of the project. We add the definition of such a product to the backlog, so that it may be worked on in the next sprint.
  \item[Only Features and Bugs in the Product Backlog] The analysis mentions that user stories used in the backlog only described features and bugs, making it difficult to formulate, prioritize, and work on technical work, such as major refactoring, and knowledge acquisition, such as investigating what library to use to solve a certain problem \cite{processanalysis}

We discuss with \group{1} and agree that we add new types of items to the backlog and specify these. A result of having these different types in the backlog is that we make it clear for project members that it is acceptable to spend time on knowledge acquisition and technical work, not only features and bugs.
\end{description}

To summarize, together with \group{1}, we decide to make the following suggestions for next semester:\todo{Overvej lige, om der er gentagelse her i forhold til det, der er nævnt lige ovenfor}

\begin{description}
  \item[Divide by Functionality] The multi-project should not be structured by subprojects as this semester, but rather every group should be able to pick any item of a common backlog. They should still use the hierarchical scrum-of-scrums structure to lessen workload on product owners. We suggest that they form new sub-projects every sprint. After each group have selected their Product Backlog Items for their Release Backlog they should be divided into a fitting number of sub-projects. All groups in each subproject should work on related parts of the project, e.g.\ if \group{A} and \group{B} are both working on Product Backlog Items which request changes in the same app, they should be part of the same subproject during that sprint. Each group must be in exactly one subproject. This approach requires greater discipline regarding continuous integration, as it is expected that groups will work closer together, as it is likely that more than one group will work on an app at a given time. In addition, it requires that groups are realistic about which user stories they commit themselves to solve.
  \item[Work on a Minimum Viable Product] Create an MVP to get the system in use quickly and satisfy the customer and thus get valuable feedback from the users.
\end{description}

We make the following changes to the development method for the final sprint in this semester:

\begin{description}
  \item[A Single Backlog] Currently, the backlog is separately managed by the subproject POs in separate lists. To make it easy to work with the backlog a single backlog should be made available. This is also necessary for the next semester to easily move onto a structure of functionality division.
  \item[New Backlog Items] We add technical work and knowledge acquisition items to the backlog to make it easy to formulate and work on those items. We specify a template for formulating user stories, and add \emph{conditions of satisfaction} for user stories. We also add \emph{constraints} to the backlog that specify constraints for one or more user stories.\todo{Nyt: læs}
\end{description}

\section{Specification of Continuous Integration}\label{sec:specsCI}
We have committed ourselves to solve a user story proposed by the developers, which states that they want a guide about continuous integration. We observe that many developers \emph{want} continuous integration (it was a high priority user story in sprint 1), but do not really understand how to use it. In the sprints up to now, we have observed developers working on features on isolated branches for several weeks without integrating it with the master branch. Likewise, only a single group has written automated tests for their code. This is a big concern, because agile development and continuous integration highly depends on automated testing. In sprint 1, we wrote a sample test for each project to make it easy for developers to start writing tests. In sprint 2, we added a code coverage measure to each project. We find, however, that this has not resulted in more tests written. Therefore, solely writing how to do does not seem to be sufficient. For this continuous integration guide, we try to argue more about why to do as we propose in order to convince developers that they should indeed follow the guidelines.

The CI guide has been released on the Redmine wiki, and is included in \appendixref{app:ci_guide}. Instructions on when to release to Artifactory has been added to the Redmine guide on dependencies. The instructions appears as specified in \sectionref{subsec:release-management}.

\section{Guide on Making Android UI Tests}\label{sec:uitestguide}
To create awareness of the opportunity for groups to have UI tests in the apps, we write a short guide on how to do this, as requested by the multi-project groups. This includes both running UI tests locally as well as in Jenkins upon each build. We do not want to spend much time on doing this, since a great amount of material is already available on the Internet. Therefore, we only find the appropriate material as well as comment on special considerations for our project. These special considerations include how to setup the database with dummy data, and faking login credentials.

The Android UI test guide can be seen in \appendixref{app:uitestguide}.