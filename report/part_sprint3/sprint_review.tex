\chapter{Sprint Review}\label{chap:sprint3_end}

\begin{chapterorganization}
  \item in \sectionref{sec:s2_goals} we evaluate how the sprint went and whether we reached our goals on a group level.
  \item in \sectionref{sec:s2_multiprj_review} we evaluate how the sprint went on the multi-project level.
\end{chapterorganization}

\section{Sprint Goals}\label{sec:s3_goals}
\begin{description}
  \item[Make Guidelines for Continuous Integration] We fulfill the desire of the developers for a clear central specification of the process. The specification is available on the Redmine Wiki, and resembles \chapterref{chap:sw_dev_method}. The specification is necessary as it is time consuming to explain it over and over, and because the process sometimes changes. A central wiki entry helps the developers keep up to date. It is to be seen if the developers actually start referring to the wiki, to settle uncertainties about the process, in their everyday work. 
  \item[Add a Guide on How to Do UI Test] We specify how to set up and run UI testing in a guide which is available on the Redmine Wiki. The guide links to the testing guide \parencite{AndroidUnit} of the Android Developer website, as it is clear and comprehensive. Our guide just adds details specific to the multi-project, e.g.\ how to initialize the database with dummy data.  
  \item[Monkey Test] In this sprint we enable monkey testing for all the apps. We create a new app for inserting dummy data into the local database, such that the apps can be run without first downloading the real database. There already exists jobs on Jenkins for monkey testing, and they are now updated to use the dummy data. We did not manage to fulfill the task of setting up a subscription based notification system for developers, as we discovered report related tasks that we prioritized more highly than this task. In any case this task is actually not necessary to fulfill the user story --- it should not have been created in the first place.
  \item[Specify the Scrum Process Used] \group{1} approached us last sprint with an analysis of the development method. The analysis raised a number of critiques, which we discuss with \group{1}. From this discussion we, in collaboration with \group{1}, revise the development method. We specify how user stories are formulated, and add new types of product backlog items.
\end{description}

\section{Multi-Project Sprint Review}\label{sec:s3_multiprj_review}
\dummy