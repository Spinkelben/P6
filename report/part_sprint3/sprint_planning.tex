\chapter{Sprint Planning}
This sprint planning proceeded ordinarily as in Scrum of Scrums. We first had a sprint planning in the subproject followed by sprint planning in the group.

\begin{chapterorganization}
  \item in \sectionref{sec:S3_bd} we describe the result, i.e.\ the selected user stories, of the \bd sprint planning meeting.
  \item in \sectionref{sec:S3_group} we describe our own group sprint planning and lists our tasks with reference to a user story.
\end{chapterorganization}

%We describe the sprint planning of our own \bd subproject. We merely attended the \gui sprint planning.

\section{\bdtitle Sprint Planning}\label{sec:S3_bd}
At the \bd sprint planning we chose to work on the following user stories in this sprint. They are labeled with a number in parentheses for reference. Notice that due to unfinished business regarding sprint 2 in this report, we have not selected big user stories. This allows us to catch up on the report writing.

\begin{description}
  \item[Make Guidelines for Continuous Integration (1)] We found out that some groups have a long-term branch that they only merge into master at sprint ends. Investigating this, people from other groups mentioned that they are not sure of how to do continuous integration. Therefore we will make some guidelines.
  \item[Add a guide on how to do UI test (2)] Even though UI testing was implemented in sprint 1, no people from other groups were aware of the facility. We will make a small guide on this.
  \item[Monkey test (3)] There still were some problems from last sprint, which we will remedy in this sprint. \todo{Elaborate on this.}
  \item[Specify the Scrum process used (4)] \todo{Reasons?}
\end{description}

\section{Group Sprint Planning}\label{sec:S3_group}
At our internal sprint planning we divided the chosen user stories into tasks and estimated those. For this sprint we have a total of 58 half days of work. \tableref{tab:sprint3_tasks} shows the tasks we have for this sprint. %Estimations in parentheses are the result of estimations being updated to due new priorities.
Tasks with a plus (+) are tasks that have been added during the sprint as they were discovered.

We made tasks for finishing the report of the previous sprint estimated for 25 half days. We estimate 11 half days for the report of this sprint. The original total amount of work estimated for both the report and user stories is 49. \todo{Write about that we're missing 9 units.}% While this is 4 units more of work than we have available, many of the report tasks are estimated as being 1, but in fact take less time. Therefore an additional 4 units of work is acceptable. This does, however, leave little to no time for writing the report of this sprint.

%The task \emph{make Jenkins job for beta releases} was also rejected during the sprint, as we discovered there was no need to fulfil this task. The Google Play group can easily ``upgrade'' an alpha release to a beta release.

\begin{table}%
  \centering
  \begin{tabular}{p{0.6\textwidth}rr}
    \toprule
    \textbf{Task} & \textbf{User Story} & \textbf{Estimation} \\
    \midrule
    Add Gradle dependency: publishToMavenLocal (+) & ? & 1 \\
    Make guide to CI best practices & 1 & 2 \\
    Make guide to UI testing & 2 & 1 \\
    Support monkey testing in all apps & 3 & 4 \\
    Setup subscription to monkey test reports in Jenkins & 3 & 4 \\
    Investigate Scrum process with \group{3} & 4 & 1 \\
    %\midrule
    %\textbf{Down-prioritized tasks} & & \\
    %\midrule
    %\midrule
    %\textbf{Missed tasks} & & \\
    %\midrule
    %\midrule
    %\textbf{Rejected tasks} & & \\
    %\midrule
    \midrule
    \textbf{Original total} & & 13 \\
    \textbf{Total} & & 13 \\
    \bottomrule
  \end{tabular}
\caption{Sprint backlog for sprint 3, excluding report tasks. The tasks are listed in no particular order.}
\label{tab:sprint3_tasks}
\end{table}