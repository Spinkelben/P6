%!TEX root = ../report.tex
\chapter{Sprint Planning}\label{chap:s3_sprintplanning}

\begin{chapterorganization}
  \item in \sectionref{sec:S3_bd} we describe the user stories we commit ourselves to during the \bd sprint planning meeting.
  \item in \sectionref{sec:S3_group} we describe the sprint planning in our group and lists our tasks with reference to a user story.
\end{chapterorganization}

\section{\bdtitle Sprint Planning}\label{sec:S3_bd}
During the \bd sprint planning we choose to work on the following user stories in this sprint. They are labeled with a number in parentheses for reference. Notice that due to unfinished report business, we do not select large user stories. This allows us to catch up on the report writing.

\begin{description}
  \item[Make Guidelines for Continuous Integration (1)] We found out that some groups have a long-term branch that they only merge into master at sprint ends. Investigating this, people from other groups mentioned that they are not sure of how to do continuous integration. Because continuous integration is an important part of our development method, we will make some guidelines on how to use it.
  \item[Add a Guide on How to Do UI Test (2)] Even though UI testing was implemented in sprint 1, no people from other groups were aware of the facility. We will make a small guide on how to do it.
  \item[Monkey Test (3)] Last sprint we did not manage to setup monkey testing. The monkey tool does not support sending extra information to apps which is required to be able to start Giraf apps. In this sprint, we will make it possible to start apps without sending extra information when running monkey tests.
  \item[Specify the Scrum Process Used (4)] Some groups have complained that the current documentation of the process used in the multi-project is too sparse. Therefore they have requested a detailed specification of the process.
\end{description}

\section{Group Sprint Planning}\label{sec:S3_group}
At our internal sprint planning we divide the chosen user stories into tasks and estimate them. For this sprint, we have a total of 55 half days of work. \tableref{tab:sprint3_tasks} shows the tasks we have committed to solve for this sprint. Tasks with a plus (+) are tasks that have been added during the sprint as they were discovered.

We make tasks for documenting the work from the previous sprint estimated for 25 half days. In addition, we estimate 11 half days for the report of sprint 3. The original total amount of work estimated for both the report and user stories is 51. This leaves 4 units of unspecified work. When we make the tasks for this sprint, we are unsure about the amount of work contained in user story 4. Thus we make the task about investigating the Scrum process with \group{1}, which then result in the additional task \emph{Make process specification}. During the sprint, we additionally add 8 half days of report tasks because we find more work. As such, we end up with a total of 59 half days.


\begin{table}%
  \centering
  \begin{tabular}{p{0.6\textwidth}rr}
    \toprule
    \textbf{Task} & \textbf{User Story} & \textbf{Estimation} \\
    \midrule
    Make guide to CI best practices & 1 & 2 \\
    Make guide to UI testing & 2 & 1 \\
    Support monkey testing in all apps & 3 & 4 \\
    Investigate Scrum process with \group{3} & 4 & 1 \\
    Make process specification (+) & 4 & 2 \\
    %\midrule
    %\textbf{Down-prioritized tasks} & & \\
    %\midrule
    \midrule
    \textbf{Missed tasks} & & \\
    Setup subscription to monkey test reports in Jenkins & 3 & 4 \\
    %\midrule
    %\textbf{Rejected tasks} & & \\
    %\midrule
    \midrule
    \textbf{Original total} & & 12 \\
    \textbf{Total} & & 14 \\
    \bottomrule
  \end{tabular}
\caption[Sprint 3 backlog]{Sprint backlog for sprint 3, excluding report tasks. The tasks are listed in no particular order.}
\label{tab:sprint3_tasks}
\end{table}