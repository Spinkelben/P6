\chapter*{Resume}
This project deals with a software project called GIRAF (Graphical Interface Resources for Autistic Folk). The project is developed by \nth{6} semester students from Software Engineering at Aalborg University. The development is a collaborative effort among groups of these students. The GIRAF software aims to assist citizens with Autism Spectrum Disorder (ASD) by easing their and their guardian's everyday activities. GIRAF provides Android apps meant for direct usage by the citizens with ASD, their guardians, and some for administrative purposes.

One of the great challenges and learning goals of this semester is to successfully work together on a software system this large. Collaboration among groups is an important part, as this eases and accelerates the development. There are 15 groups with 1--4 students, working in \emph{Scrum of Scrum}. The development on the software system is split into four sprints, each with a sprint planning and a sprint review meeting. At the sprint planning meeting, each group selects a fitting number of items from the product backlog, which they complete during the sprint. This group, sw609f15, has primarily worked on continuous integration and the build environment supporting the software system. And as such we have selected product backlog items related to this.

\todo{hvad lavede vi i sprint 1, 2, 3, og 4?}

\todo{Hvad havde vi af collaborations?}

\todo{kort konklusion}