\chapter*{Resume}
Graphical Interface Resources for Autistic Folk (GIRAF) is a software suite developed by \nth{6} semester students from Software Engineering at Aalborg University. The development is a collaborative effort among groups of these students. The GIRAF software aims to assist citizens with Autism Spectrum Disorder (ASD) by easing their and their guardian's everyday activities. GIRAF provides Android apps meant for direct usage by the citizens with ASD, their guardians, and some for administrative purposes.

One of the great challenges and learning goals of this semester is to successfully work together on a software system this large. Collaboration among groups is an important part, as this eases and accelerates the development. There are 15 groups with 1--4 students, working in \emph{Scrum of Scrums}. The development on the software system is split into four sprints, each with a sprint planning and a sprint review meeting. At the sprint planning meeting, each group selects a fitting number of items from the product backlog, which they complete during the sprint. We, group sw609f15, are responsible for the overall software development method. We primarily work on continuous integration and the build environment supporting the software system.

In sprint 1, we set up the basic automated build on Jenkins. This includes running automated tests and generating documentation nightly. Additionally, we discuss various Git merging strategies related to continuous integration.

In sprint 2, we add a number of automated tests to compensate for missing unit tests in the individual projects, including monkey testing. We make the continuous integration server create and publish test coverage reports for developers. We change the way internal dependencies are management together with the Git-responsible group by replacing error-prone Git submodules with a Maven repository.

In sprint 3, we evaluate and refine the development method based on the experiences we have had during the previous sprints. We do this in collaboration with another group. In addition, we make monkey tests use a local test database with dummy data to avoid downloading the full database.

In sprint 4, we further improve the test environment by replacing the emulator with physical devices, enabling us to test on a wider range of devices. This also decreases the overall time it takes to build project.

In conclusion, we have significantly improved the build environment --- in particular the continuous integration platform. The developers now have a simpler-to-use development environment, which aids in improving the stability of the software. In addition, we have defined and refined the development method followed by all groups in the development GIRAF. The GIRAF project is in a significantly more stable state than before, and there is now a uniform experience across all apps.
