Continuous integration of a complex software suite is challenging. We present a configuration of continuous integration and development method for a large team of \nth{6} semester Software Engineering students at Aalborg University working on the GIRAF Android apps aimed at easing the lives of people with Autism Spectrum Disorders and their caretakers. The project is organized according to Scrum of Scrums, comprising 14 groups of 1--4 persons. The code base is inherited from previous semesters. We setup a tool, Jenkins, to automatically build and test apps and libraries that are developed by the project, whenever changes are made. The apps are automatically tested on physical tablets wirelessly and uploaded to Google Play. Libraries are automatically published to a Maven repository saving all built versions. We run automated monkey tests on all GIRAF apps, filling the database with dummy data with an app before starting and blocking the Android notification bar to avoid changing settings. We have significantly improved the build environment compared to what we started with.