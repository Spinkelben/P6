Continuous integration and deployment in a complex software system poses many challenges. We present a configuration of continuous integration and a development method for a large team of \nth{6} semester Software students at Aalborg University working on the GIRAF app suite. The suite aims at easing the lives of people with Autism Spectrum Disorders and their caregivers. The project is organized according to Scrum of Scrums, comprising 15 groups of 1--4 students. The code base is inherited from previous semesters. The caretakers require a working product. We setup a tool, Jenkins, to automatically build and test apps and libraries that are developed for the project whenever changes are made to the code. The apps are automatically tested on physical tablets wirelessly and deployed to Google Play. Libraries are published to a Maven repository. Monkey tests are run on all GIRAF apps. We significantly improve the build environment compared to what we start with. The development method we specify together with the improved build environment enable groups to deliver what the stakeholders want.