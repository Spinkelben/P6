Continuous integration and deployment in a complex software system pose many challenges. We present a configuration of continuous integration and a development method for a large team of \nth{6} semester Software students at Aalborg University working on the GIRAF app suite. The suite aims at easing the lives of people with Autism Spectrum Disorders and their caretakers. The project is organized according to Scrum of Scrums, comprising 15 groups of 1--4 students. The code base is inherited from previous semesters. The caretakers require a working product. We setup a tool, Jenkins, to automatically build and test apps and libraries that are developed for the project whenever changes are made to the code. The apps are automatically tested on physical tablets wirelessly and deployed to Google Play. Libraries are automatically published to a Maven repository. Monkey tests are automatically run on all GIRAF apps. We have significantly improved the build environment compared to what we started with. The development method we have specified together with the improved build environment has enabled groups to deliver what the stakeholders want.