\chapter{UI Test Guide}\label{app:uitestguide}
UI tests can test various UI actions and that those actions result in the correct result. For example, a test can click on a button and assert that the correct view opens.

An introduction on how to do UI testing can be found \href{https://developer.android.com/training/testing/ui-testing/espresso-testing.html\#setup}{here}\footnote{\url{https://developer.android.com/training/testing/ui-testing/espresso-testing.html\#setup}}.

As the introduction mentions, it is important to add the following to \mono{build.gradle} of the app being tested:

\begin{gradlecode}[]
android {
    defaultConfig {
        testInstrumentationRunner "android.support.test.runner.AndroidJUnitRunner" 
    }
    packagingOptions {
        exclude 'LICENSE.txt'
    }
}

dependencies {
    androidTestCompile 'com.android.support.test:testing-support-lib:0.1'
    androidTestCompile 'com.android.support.test.espresso:espresso-core:2.0'
}
\end{gradlecode}

Due to the various Giraf apps requiring extra information when starting and a database, some extra work has to be done. To demonstrate this, an example UI test for Launcher can be found in:

\begin{lstlisting}[breakatwhitespace=false, numbers=none]
launcher/launcher-app/src/androidTest/java/dk/aau/cs/giraf/launcher/test/HomeActivityUITest.java
\end{lstlisting}

The test class extends the \mono{ActivityInstrumentationTestCase2} class. The activity that one wishes to test must be in the \mono{<>}. The following snippet tests the \mono{HomeActivity}.

\begin{javacode}
public class HomeActivityUITest
        extends ActivityInstrumentationTestCase2<HomeActivity>
\end{javacode}

The method \mono{setUp} is run before each test. In this case it creates an intent with a guardian id to start the \mono{HomeActivity} of the launcher. To do this a helper is created that creates some dummy data.
\begin{javacode}
@Before
public void setUp() throws Exception {
    super.setUp();
    injectInstrumentation(InstrumentationRegistry.getInstrumentation());

    Intent intent = new Intent();
    intent.putExtra(Constants.GUARDIAN_ID, 1);
    setActivityIntent(intent);

    Helper h = new Helper(getInstrumentation().getTargetContext().getApplicationContext());
    h.CreateDummyData();

    mActivity = getActivity();
}
\end{javacode}

When each test has finished the database is cleared.
\begin{javacode}
@After
public void tearDown() throws Exception {
    super.tearDown();
    Helper h = new Helper(getInstrumentation().getTargetContext().getApplicationContext());
    h.clearTables();
}
\end{javacode}

The actual test itself clicks on the settings button on the home screen and checks that the window opens.
\begin{javacode}[breakatwhitespace=false]
public void testSettingsButton() {
    onView(withId(R.id.settings_button)).perform(click());
    onView(withId(R.id.settingsListFragment)).check(ViewAssertions.matches(isDisplayed()));
}
\end{javacode}

The test can then be run by running the \mono{connectedCheck} target.