\chapter{Git Hooks}\label{app:git_hook}
\section{Post-receive hook}
\label{sec:app_git_hook_post}

\begin{pythoncode}[caption=Post-receive git hook for managing library releases and triggering Jenkins]
#!/usr/bin/python

import os
import subprocess
import re
import fileinput
import requests
from requests.exceptions import ConnectionError
from requests.exceptions import Timeout

VERSION_FILE = '/var/lib/jenkins/project_version_codes/libversion'
JENKINS_BASE_URL = 'http://localhost/jenkins'
GIT_RO_BASE_URL = 'http://cs-cust06-int.cs.aau.dk/git-ro'

class Library(object):
    def __init__(self, name, major, minor, patch, jobname):
        self.name = name
        self.major = major
        self.minor = minor
        self.patch = patch
        self.jobname = jobname

def main():
    """Main function for handling push"""
    # Load libraries
    libraries = load_libraries()
    # Read repo name
    repo = re.sub('.git', '', execute_sh_cmd('basename $(pwd)')[0])
    # If repo is not defined as a library, just trigger Jenkins the regular way.
    if not is_library(libraries, repo):
        try:
            trigger_jenkins_poll(repo)
            print "Thank you for your push. Jenkins will be serving you in a moment."
        except (ConnectionError, Timeout) as e:
            print "ERROR TRIGGERING BUILD"
            print e
        return

    # Read from stdin
    for line in fileinput.input():
        old_rev, new_rev, refs = line.split()
        # Only deploy if on master branch
        if not 'master' in refs:
            continue
        recent_msg = None
        # Get commits in push
        commit_ids = get_commit_ids(old_rev, new_rev)
        for commit in commit_ids:
            commit_msg, exit_code = execute_sh_cmd('git log --format=%%B -n 1 %s' % (commit,))
            if commit == new_rev:
                if exit_code == 0:
                    recent_msg = commit_msg
                continue
            if exit_code == 0:
                trigger_build(commit_msg, commit, libraries, repo)

        if recent_msg != None:
            trigger_build(recent_msg, new_rev, libraries, repo, True)

    write_libraries_to_disc(libraries)

def is_library(libraries, repo_name):
    has_lib = False
    for lib in libraries:
        if repo_name == lib.name:
            has_lib = True
            break
    return has_lib

def load_libraries():
    """Loads all known libraries and their current versions from the property file."""
    libraries = []
    if os.path.isfile(VERSION_FILE):
        with open(VERSION_FILE) as v:
            # Regex for matching lib name, major version, minor version, and Jenkins job name
            rg = re.compile('((?:\w*-*)+), major=(\d*), minor=(\d*), patch=(\d*), jobname=((?:\w*-*)+)', re.IGNORECASE|re.DOTALL)
            libs = v.readlines()
            for lib in libs:
                m = rg.search(lib)
                if m:
                    libraries.append(Library(m.group(1), m.group(2), m.group(3), m.group(4), m.group(5)))
            v.close()
    return libraries

def write_libraries_to_disc(libraries):
    """Writes the libraries back to the libary file"""
    # Read updated versions
    versionFile = open(VERSION_FILE, 'wr+')
    versionFile.truncate()
    for repo in libraries:
        versionFile.write('repo=%s, major=%s, minor=%s, patch=%s, jobname=%s\n' % (repo.name, repo.major, repo.minor, repo.patch, repo.jobname))
    versionFile.close()

def execute_sh_cmd(cmd):
    """Executes the given command

    Return value: Tuple containing (output, exit code).
    """
    p = subprocess.Popen(cmd, shell=True, stdout=subprocess.PIPE)
    val = p.stdout.read().rstrip('\n')
    exit_code = p.wait()
    return val, exit_code

def get_commit_ids(start_id, end_id):
    """Returns commit ids from the given interval, from first to last"""
    output, exit_code = execute_sh_cmd('git rev-list --reverse %s..%s' % (start_id, end_id))
    if exit_code != 0 or output == "":
        return []

    # Parse output
    return output.split('\n')


def print_version_name(version_name):
    """Pretty-prints the version name."""
    print "-------------"
    print "RELEASE: %s" % version_name
    print "-------------"

def trigger_jenkins_poll(repo_name):
    """Sends a poll build request to Jenkins."""
    url = "%s/git/notifyCommit" % (JENKINS_BASE_URL)
    payload = {'url': "%s/%s" % (GIT_RO_BASE_URL, repo_name)}
    requests.get(url, params=payload)

def send_build_request(job_name, version_name, commit_id):
    """
    Sends a build POST request to Jenkins.

    Parameters:
        job_name:     The name of the Jenkins job building this library.
        version_name: The version name of the library.
        commit_id:  The corresponding commit id.
    """
    url = "%s/view/Libraries/job/%s/buildWithParameters" % (JENKINS_BASE_URL, job_name)
    payload = {'libVersion': version_name, 'commitId': commit_id}
    requests.post(url, data=payload)

def trigger_build(commit_msg, commit_id, libraries, repo_name, publish_snapshot=False):
    """
    Triggers a build on Jenkins.

    Parameters:
        commit_msg:       The contents of the commit message. Used for versioning.
        commit_id:        The id of the commit to build.
        libraries:        The known libraries and their versions.
        repo_name:        The name of the repository.
        publish_snapshot: If true, snapshots will be published.
    """
    version_name = ""
    for lib in libraries:
        if lib.name == repo_name and commit_msg != None:
            new_major = lib.major
            new_minor = lib.minor
            new_patch = lib.patch
            if "@major" in commit_msg.lower():
                new_major = str(int(lib.major) + 1)
                new_minor = "0"
                new_patch = "1"
                version_name = "%s.0.0" % (new_major,)
            elif "@minor" in commit_msg.lower():
                new_minor = str(int(lib.minor) + 1)
                version_name = "%s.%s.0" % (lib.major, new_minor)
                new_patch = "1"
            elif "@patch" in commit_msg.lower():
                version_name = "%s.%s.%s" % (lib.major, lib.minor, lib.patch)
                new_patch = str(int(lib.patch) + 1)
            else:
                if not publish_snapshot:
                    return
                version_name = "%s.%s.%s-SNAPSHOT" % (lib.major, lib.minor, lib.patch)
            # Send request
            try:
                send_build_request(lib.jobname, version_name, commit_id)
                # Apply lib changes (request succeeded)
                lib.major = new_major
                lib.minor = new_minor
                lib.patch = new_patch
                print_version_name(version_name)
            except (ConnectionError, Timeout) as e:
                print "ERROR TRIGGERING BUILD"
                print e
            return

if __name__ == "__main__":
    main()
\end{pythoncode}
\section{Pre-receive hook}
\label{sec:app_git_hook_pre}
\begin{pythoncode}[caption=Pre-receive git hook for checking for common mistakes]
#!/usr/bin/python
import fileinput
import subprocess
import os.path

GIT_DISABLED_FILE = "/var/lib/jenkins/git_disabled"

def main():
    if not check_jenkins_enabled():
        print "Jenkins is busy at the moment. Please try again soon."
        exit(1)
    # Read from stdin
    for line in fileinput.input():
        old_rev, new_rev, _ = line.split()
        # Check if contains snapshot
        check_snapshot(old_rev, new_rev)
        # Check if contains dynamic version reference
         check_dynamic_version(old_rev, new_rev)
        # Check if DatabaseCredentials is added
        check_database_credentials(old_rev, new_rev)

def check_snapshot(old_rev, new_rev):
    """Checks if a snapshot has been added"""
    _, exit_code = execute_sh_cmd('git diff %s %s | grep + | grep 0.0-SNAPSHOT' % (old_rev, new_rev))
    if exit_code == 0:
        print "Jenkins is dissapointed:"
        print "It looks like you are referencing a 0.0-SNAPSHOT."
        print "Please fix this and commit again."
        print "\nCommit hash: %s" % (new_rev,)
        exit(1)

def check_dynamic_version(old_rev, new_rev):
    """Checks if a dynamic version has been added"""
    _, exit_code = execute_sh_cmd('git diff %s %s -- \'*.gradle\' | grep + | grep "[[:digit:]]\.+"' % (old_rev, new_rev))
    if exit_code == 0:
        print "Jenkins is dissapointed:"
        print "It looks like you are referencing a library using a dynamic version number (e.g. 2.2.+)"
        print "Please fix this and commit again."
        print "\nCommit hash: %s" % (new_rev,)
        exit(1)

def check_database_credentials(old_rev, new_rev):
    """Checks if database credentials file has been added"""
    diff_files, _ = execute_sh_cmd('git diff --name-only %s %s' % (old_rev, new_rev))
    for new_file in diff_files:
        if 'dk/aau/cs/giraf/localdb/DatabaseCredentials.java' in new_file:
            print "Jenkins is dissapointed:"
            print "It looks like you have commited the DatabaseCredentials file."
            print "Please remove the file, clear your history and commit again."
            print "\nCommit hash: %s" % (new_rev,)
            exit(1)

def check_jenkins_enabled():
    """Checks if git is enabled"""
    if not os.path.isfile(GIT_DISABLED_FILE):
        return True
    f = open(GIT_DISABLED_FILE, 'r')
    content = f.read()
    if "1" in content:
        return False
    else:
        return True

def execute_sh_cmd(cmd):
    """Executes the given command

    Return value: Tuple containing (output, exit code).
    """
    p = subprocess.Popen(cmd, shell=True, stdout=subprocess.PIPE)
    val = p.stdout.read().rstrip('\n')
    exit_code = p.wait()
    return val, exit_code


if __name__ == "__main__":
    main()
\end{pythoncode}