\chapter{Wireless Router Setup}\label{app:router_setup}

\section{Login Information}
\begin{tabular}{ll}
Wireless SSID & MariusTestNetworkTM\\
Management & \mono{http://192.168.1.1}\\
&\mono{http://172.25.11.91}\\
&\mono{SSH root@192.168.1.1}\\
Username & \mono{root}\\
Password & \mono{routeradmin}\\
\end{tabular}

\section{IPv4 WAN configuration}
Applying the following IPv4 WAN settings will (in combination with the router's MAC address) put the router on a network visible to the server.

\vspace{.4cm}
\noindent\begin{tabular}{ll}
\toprule
Name & Value\\
\midrule
Type & Static\\
Address & \mono{172.25.11.91}\\
Netmask & \mono{255.255.255.0}\\
Gateway & \mono{172.25.11.1}\\
DNS1 & \mono{172.18.21.2}\\
DNS2 & \mono{172.18.21.34}\\
\bottomrule
\end{tabular}

\vspace{.5cm}
\noindent However, a change in the network setup at Aalborg University is still needed. Contact Per Majdal, IT Services in Cluster 3. He can enable a specific LAN socket (for example in the group rooms) access to the server network. Bring him the MAC address of the router (important: see this in software, the sticker on the device has the \emph{wrong} MAC address) and the LAN socket name (sticker next to the socket, e.g. \mono{K6.41 F10}).

\vspace{1cm}
\noindent The scripts are located in \mono{/www/cgi-bin/} and \mono{/bin/}.