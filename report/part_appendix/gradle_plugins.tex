\chapter{Gradle Plugins}\label{app:gradle_plugins}
The plugin to help publish APKs can be seen in \listingref{lst:gradle_plugin_publish_app}. The plugin to publish libraries can be seen in \listingref{lst:gradle_plugin_library_app}.

\begin{gradlecode}[caption=Gradle Plugin to help publish APKs (written in Groovy),label=lst:gradle_plugin_publish_app,breakatwhitespace=false]
package dk.giraf.gradle

import org.gradle.api.Plugin
import org.gradle.api.Project
import org.gradle.api.GradleException
import java.util.regex.Pattern

class DeployPlugin implements Plugin<Project> {

  def manifestFilePath = 'src/main/AndroidManifest.xml'
  def versionCodesFilePath = '/srv/jenkins/project_version_codes/versionCodes.properties'
  def keyStorePath = '/srv/jenkins/google_play_keys/keystore.properties'

  void apply(Project project) {
    //Create play account extension
    project.extensions.create ('playAcc', GooglePlayAccount)

    project.task('renameApk') << {
      // Set custom apk name
      project.android.applicationVariants.all { variant ->
        def versionCode = getVersionCode(project, applicationId)['value']

        def applicationId = project.android.applicationVariants.applicationId[0]
        def name
        variant.outputs.each { output ->
          def apkDirectory = output.outputFile.parentFile
          def manifestParser = new com.android.builder.core.DefaultManifestParser()

          if (output.zipAlign) {
            name = applicationId + "_v" + manifestParser.getVersionName(project.android.sourceSets.main.manifest.srcFile) + "b" + versionCode + "_" + variant.buildType.name.toLowerCase() + "_aligned.apk"
            output.outputFile = new File(apkDirectory, name)
          }

          name = applicationId + "_v" + manifestParser.getVersionName(project.android.sourceSets.main.manifest.srcFile) + "b" + versionCode + "_" + variant.buildType.name.toLowerCase() + "_unaligned.apk"
          output.packageApplication.outputFile = new File(apkDirectory, name)
        }
      }
    }
    // Autoincrement version code in android manifest
    project.task('increaseVersionCode') << {
      def applicationId = project.android.applicationVariants.applicationId[0]
      // Open version codes property file
      if (project.file(versionCodesFilePath).exists() != true) {
        throw new GradleException("No version code file found. Only jenkins should run this task")
      }

      // Open manifest file and find versionCode
      def manifestFile = project.file(manifestFilePath)
      def pattern = Pattern.compile("versionCode=\"(\\d+)\"")
      def manifestText = manifestFile.getText()
      def matcher = pattern.matcher(manifestText)
      matcher.find()

      def newVersion = getVersionCode(project, applicationId)
      def versionCode = newVersion['value']
      if (!newVersion['created']) {
        // Increment version code if not new
        versionCode++
        // Write version code to manifest
        def manifestContent = matcher.replaceAll("versionCode=\"" + versionCode + "\"")
        manifestFile.write(manifestContent)
        // Increment version code from version codes properties file
        def versionCodesFile = project.file(versionCodesFilePath)
        def versionPattern = Pattern.compile(applicationId + "=(\\d+)")
        def versionCodesText = versionCodesFile.getText()
        def versionMatcher = versionPattern.matcher(versionCodesText)
        versionMatcher.find()
        def newVersionCodesText = versionMatcher.replaceAll(applicationId + "=" + versionCode)
        versionCodesFile.write(newVersionCodesText)
      }
    }

    //Make release config generation depend on renaming
    project.tasks.whenTaskAdded { task ->
      if (task.name == 'generateReleaseBuildConfig') {
        task.dependsOn 'renameApk'
      }
    }

    // Set keystore path
    project.android {
      signingConfigs {
        release {
          Properties props = new Properties()
          if (project.rootProject.file(keyStorePath).exists())
          {
            props.load(new FileInputStream(project.rootProject.file(keyStorePath)))
            storeFile project.rootProject.file(props['storeFile'])
            storePassword props['storePassword']
            keyAlias props['keyAlias']
            keyPassword props['keyPassword']
          }
        }
      }
    }

    project.android {
      buildTypes {
        release {
          signingConfig signingConfigs.release
          zipAlignEnabled = true
        }
      }
    }

    project.android.dexOptions.preDexLibraries = !project.getRootProject().hasProperty('disablePreDex')
  }

  // Creates a new app in the version code file
  def makeNewApp(manifestFile, versionCodesFile, manifestMatcher, applicationId) {
    def versionCode = 1

    // Write version code to manifest
    def manifestContent = manifestMatcher.replaceAll("versionCode=\"" + versionCode + "\"")
    manifestFile.write(manifestContent)

    // Add version code to version codes properties file
    def versionCodesText = versionCodesFile.getText()
    def newVersionCodesText = versionCodesText + "\n" + applicationId + "=" + versionCode
    versionCodesFile.write(newVersionCodesText)
  }

  def getVersionCode(project, applicationId) {
    // Load version codes properties
    if (project.rootProject.file(versionCodesFilePath).exists()) {
      Properties versionsProp = new Properties()
      versionsProp.load(new FileInputStream(project.rootProject.file(versionCodesFilePath)))
      // If version code already exists
      if (versionsProp[applicationId] == null) {
        // no version code was found for this project, make a new one!
        def versionCodesFile = project.file(versionCodesFilePath)
        def manifestFile = project.file(manifestFilePath)
        def pattern = Pattern.compile("versionCode=\"(\\d+)\"")
        def manifestText = manifestFile.getText()
        def matcher = pattern.matcher(manifestText)
        makeNewApp(manifestFile, versionCodesFile, matcher, applicationId)
        return [value: 1, created: true]
      } else {
        // Load version code. Increment as the incrementation task is run after thus task
        return [value: Integer.parseInt(versionsProp[applicationId]), created: false]
      }
    }
    else {
      return [value: 1, created: true]
    }
  }
}

class GooglePlayAccount {
  String accountEmail = '845822318110-g2fur4jtaa6g1jek6n4m5i3jbpmfi6em@developer.gserviceaccount.com'
  String keyPath = '/var/lib/jenkins/google_play_keys/google_play_api_key.p12'
}
\end{gradlecode}

\begin{gradlecode}[caption=Gradle Plugin to publish libraries (written in Groovy),label=lst:gradle_plugin_library_app,breakatwhitespace=false]
package dk.giraf.gradle

import org.gradle.api.Plugin
import org.gradle.api.Project
import org.gradle.api.GradleException
import org.gradle.api.publish.maven.MavenPublication

class LibraryPlugin implements Plugin<Project> {

  def artifactUrlSnapshots = 'http://cs-cust06-int.cs.aau.dk/artifactory/libraries-snapshots'
  def artifactUrlReleases = 'http://cs-cust06-int.cs.aau.dk/artifactory/libraries-releases'
  def artifactCredentialsPath = '/srv/jenkins/credentials/artifactory.properties'

  void apply(Project project) {
    // Create giraf plugin extension
    GirafLibrary girafLib = project.extensions.create ('girafLibrary', GirafLibrary)

    // Set version name
    if(!project.hasProperty('versionName')) {
      project.ext.set('versionName', '0.0-SNAPSHOT')
    }
    // Apply plugins needed for libraries
    project.afterEvaluate {
      // Read artifactory credentials
      Properties props = new Properties()
      if (project.rootProject.file(artifactCredentialsPath).exists()) {
        props.load(new FileInputStream(project.rootProject.file(artifactCredentialsPath)))

        project.publishing {
          publications {
            repositories.maven {
              if(project.versionName.endsWith('-SNAPSHOT')) {
                url artifactUrlSnapshots
              } else {
                url artifactUrlReleases
              }
              credentials {
                username props['username']
                password props['password']
              }
            }
          }
        }
      }
      // Add publication even when no credentials file exists. It is used for local publishing.
      project.publishing {
        publications {
          maven(MavenPublication) {
            groupId 'dk.aau.cs.giraf'
            artifactId project.girafLibrary.libraryName
            version project.versionName
            artifact 'build/outputs/aar/library.aar'
          }
        }
      }
      // Add rename dependency
      project.tasks.findByName('generateReleaseBuildConfig').dependsOn(project.tasks.findByName('renameLib'))

      project.tasks.findByName('publishToMavenLocal').dependsOn project.tasks.findByName('build')

      project.android.dexOptions.preDexLibraries = !project.getRootProject().hasProperty('disablePreDex')
    }

    project.task('renameLib') << {
      // Set library output name
      project.android.libraryVariants.all { variant ->
        variant.outputs.each { output ->
          def outputFile = output.outputFile
          if (outputFile != null && outputFile.name.endsWith('.aar')) {
            def fileName = "library.aar"
            output.outputFile = new File(outputFile.parent, fileName)
          }
        }
      }
    }
  }
}

class GirafLibrary {
  String libraryName
}
\end{gradlecode}