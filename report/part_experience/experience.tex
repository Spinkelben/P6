\chapter{Evaluation of the Multi-Project}%\label{chap:...}
A chapter documenting the experiences gained from working in a multi-project. Reflections
and knowledge to be passed on to future students. \todo{Describe how there are several layers of evaluations. Technical work -> process in our group -> multi-project -> process of the multi-project}

\todo{Place this somewhere: Homogenous project og vi har ikke tilført mange nye features}

\begin{chapterorganization}
  \item in \sectionref{conc:userstories} we evaluate the technical work we have done this semester in terms of what user stories we selected and completed;
  \item in \sectionref{conc:multi_project_eval} we evaluate the multi-project process
\end{chapterorganization}

\todo{Hvor evalueres vores egen interne scrum process i gruppen?}

\section{Evaluation of Group Work on User Stories}\label{conc:userstories}
\todo{Kort: Hvad har vi lavet?}

\todo{Fyldestgørende status på det vi har arbejdet med (jenkins mm.)}

\todo{Beskriv at udviklingsmiljøet mm.\ er meget bedre nu.}

\section{Evaluation of Multi-Project Process}\label{conc:multi_project_eval}
\todo{Opsummere hvordan procesen har kørt, hvad der var godt og hvad der var skidt. Læg vægt på at processen løbende er blevet evalueret.}

\todo{Tilvænning, var der noget der overraskede os ved at arbejde så mange sammen?}

\todo{status meetings}

\todo{Redmine? Vi bruger det ikke rigtigt}



\chapter{Recommendations for Future Developers}
\todo{kapitel til alt hvad vi syntes de næste skal vide hvis de ikke har læst resten af rapporten}

\todo{De bør få et ordentligt testmiljø op!}

\section{Process Recommendations}
\todo{Hvordan de bør styre procesen næste år. Anbefalet en samlet, unified backlog!}

\section{Recommandations for Jenkins}
\todo{Hvad der kan gøres for at forbedre jenkins fremover. Referer til den nuværende jenkin struktur appendix}
For future students: How to make new libraries published (server file /srv/jenkins/project\_version\_codes/libversion)
