\chapter{Evaluation of the Multi-Project}\label{chap:evaluation}
\todo{Chapter introduction}
The overall goal of this project was to make the GIRAF apps more usable and refined than when we overtook it from last year's students. This goal has been apparent throughout the different parts of the project: From the individual technical items the groups have worked on, to the refinements made to the development method with build automation and continuous integration. Our personal goal was to improve build automation and the overall testing facilities of the GIRAF project. In this chapter, we will evaluate the project with respect to these goals.

There are several layers to evaluate: The technical work, i.e.\ the user stories we completed as well as other tasks we completed by necessity; the development process followed in the group; the accomplishments of the multi-project as a whole, i.e.\ all individual group's collective progress; and finally the process surrounding the entire multi-project. \todo{Maybe a figure to illustrate the structure of this, going from the inner-most layer to outer-most?}

\begin{chapterorganization}
  \item in \sectionref{conc:userstories} we evaluate the technical work we have done this semester in terms of what user stories we selected and completed;
  \item in \sectionref{conc:internalprocess} we evaluate the development method followed internally in our group;
  \item in \sectionref{conc:multi_project_eval} we evaluate the collective progress towards the goal of evolving the GIRAF project into something useful, helping autistic citizens;
  \item in \sectionref{conc:multi_project_process_eval} we evaluate the development method employed across the multi-project.
\end{chapterorganization}

\section{Group Work on User Stories}\label{conc:userstories}
During this project we have accomplished a number of backlog items. The result of all this work is that the build environment, the continuous integrations platform, Jenkins, in particular, is significantly improved from what we inherited. When we started the continuous integration platform was just a regular build platform. We changed the configuration such that the platform automatically builds a new version whenever changes are pushed to the master branch of a repository. We also run any unit and UI tests after the build process, and we only consider the build successful if all the tests pass. If the build was successful we also publish the new version. If it is an app, it is published on the alpha track on the Google Play Store. If it is a library the commit message decides whether to publish a major or minor release, a patch or a development snapshot. Before all these changes, developers manually had to start the builds on Jenkins and no test were run. We have enabled email notifications so that whenever a build fails, the developer responsible for the bad build gets a notification email with a link to the failed build. Furthermore we provide statistics on successful builds. We provide statistics like code coverage, and lint errors and warnings. The automation of the build process gives quick feedback when errors occur and has helped greatly with heightening the overall stability of the apps, and provides the confidence that it requires to push changes to the master branch often.

There are very few tests of the apps in general, so we run monkey tests every night, where the newest version of every app is tested. If an app crashes during a test, a stacktrace is available as well as a screen capture of the app, captured as the app crashed. The monkey tests were only really functioning quite late in the project and therefore have had limited impact on the overall stability of the apps.

We have worked on several backlog items which requested that the builds were faster. We have succeeded in speeding up the build every time, but we still end up with slower builds than when we started. This is because we do much more work now when we build, than what was done before.

We also have worked an different tasks which would make life easier for the developers. In collaboration with the Git group we have removed submodules from Git, which were a hassle for the developers to work with. Instead we have libraries which are pre-compiled and the different versions are specified in the build script. The developers also wanted a easy way of installing the newest version of all apps on a tablet. We have developed a script which accomplishes this task.

In addition to Jenkins, we also have responsibility for the development process in general. While we handle these responsibilities we gather information which are relevant for some or all of the other groups in the multi-project. To make this information available we use the wiki on Redmine. We have made many guides on different topics including UI testing, unit testing, multi-project development process, and continuous integration.

As a whole, the developers now work in a simpler to use environment, which provides many services that improve confidence in the stability of the software.

\section{Internal Development Method}\label{conc:internalprocess}
In our group we have used a physical sprint backlog containing all the tasks to complete in a sprint. These tasks are estimated. In combination with a \emph{burndown} chart we have been able to track our progress throughout a sprint and prioritize tasks. The Daily Scrum meeting has ensured that we consistently updated our sprint backlog and burndown chart. Even though there were times where we got behind schedule, we have been able to adapt and Scrum has worked well for us. The method also fit nicely with the scrum of scrum process used on the multi-project level, especially the common language of backlog items like user stories have been very useful.

\section{The Development of GIRAF}\label{conc:multi_project_eval}

\todo{Place this somewhere: Homogenous project og vi har ikke tilført mange nye features. Kunde til sprint review: Apps ser ens ud og det vi har bedt om}

\section{Multi-Project Development Method}\label{conc:multi_project_process_eval}
The multi-project has been organized in a Scrum-fashion \todo{Scrum of Scrums?}. We have continually throughout all sprints refined the method to improve our process. What is written in \chapterref{chap:sw_dev_method} is the result of refining the Scrum method to suit our needs. The work we have put into formalizing the development method and the following refinement of it has freed the other groups to dedicate their time to completing their user stories related to more practical work on the GIRAF project. As such, we have enabled the \gui and \db groups to reach the GIRAF project goals in a more efficient way, and thus overall accelerate the development in these groups.

We found initiating the project more difficult than our previous projects, because the large size of the multi-project added a new complexity we had not encountered before. It took some acclimatization to work in a large software project, and we faced many challenges of getting things to run relatively smoothly. \todo{Beskriv mere specifikt hvad vi har lært af at arbejde så mange sammen}

In sprint 1 and to a lesser degree sprint 2, there was some confusion among the groups regarding the development method. We believe this led to the problem with the duration of the weekly status meetings, because the development method was not clearly defined at this point, and we had to spend time clarifying this. \todo{Har vi overhovedet beskrevet de lange møder tidligere?}

At each sprint planning meeting, the requirements of the external customers have been mostly diffuse. This indicate that a agile development method is suitable. A factor contributing to this choice is the fact that no developer this semester knew anything about the GIRAF project beforehand, and we still had to make progress in a short amount of time.

\todo{agil process, samarbejde, det er ikke nødvendigt at formalisere samarbejdet -> godt tegn, collaboration}
\todo{Håndtering af backlog items.}
