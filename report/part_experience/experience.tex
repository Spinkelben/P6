\chapter{Evaluation of the Multi-Project}\label{chap:evaluation}
\todo{Chapter introduction}

There are several layers to evaluate: The technical work, i.e.\ the user stories we completed as well as other tasks we completed by necessity; the development process followed in the group; the accomplishments of the multi-project as a whole, i.e.\ all individual group's collective progress; and finally the process surrounding the entire multi-project. \todo{Maybe a figure to illustrate the structure of this, going from the inner-most layer to outer-most?}

\begin{chapterorganization}
  \item in \sectionref{conc:userstories} we evaluate the technical work we have done this semester in terms of what user stories we selected and completed;
  \item in \sectionref{conc:internalprocess} we evaluate the development method followed internally in our group;
  \item in \sectionref{conc:multi_project_eval} we evaluate the collective progress towards the goal of evolving the GIRAF project into something useful, helping autistic citizens;
  \item in \sectionref{conc:multi_project_process_eval} we evaluate the development method employed across the multi-project.
\end{chapterorganization}

\section{Group Work on User Stories}\label{conc:userstories}

During this project we have accomplished a number of backlog items. The result of all this work is that the build environment, the continuous integrations server in particular, is significantly improved from what we inherited. When we started the continuous integration server was just a regular build server. We changed the configuration such that the server automatically builds the new version whenever changes are pushed to the master branch. We also run any unit and UI tests after the build process, and we only consider the build successful if all the tests pass. If the build was successful we also publish the new version. If it is an app, it is published on the alpha track on the Google Play Store, if it is an library the commit message decides whether to publish a major or minor release, a path or a development snapshot. Before all these changes, developers manually had to start the builds on the server and no test were run. We have enabled email notifications so whenever a build fails, the developer responsible for the bad build gets a notification email with a link to the failed build.
 
Automated build
Automated test
Email notification
Auto upload of Alpha releases
Monkey, Screenshot
Faster build
Install all apps
code coverage
Libraries
Guide UI, Android Test, Process, CI, Jenkins
Library Priority
Debug Test
Download and Install all apps Local

\todo{Kort: Hvad har vi lavet?}
\todo{Fyldestgørende status på det vi har arbejdet med (jenkins mm.)}
\todo{Beskriv at udviklingsmiljøet mm.\ er meget bedre nu.}

\section{Internal Development Method and Process}\label{conc:internalprocess}
\todo{Var vores scrum god?}

\section{Multi-Project as a Whole}\label{conc:multi_project_eval}
\todo{Place this somewhere: Homogenous project og vi har ikke tilført mange nye features}

\section{Multi-Project Process}\label{conc:multi_project_process_eval}
\todo{Opsummere hvordan procesen har kørt, hvad der var godt og hvad der var skidt. Læg vægt på at processen løbende er blevet evalueret.}

\todo{Tilvænning, var der noget der overraskede os ved at arbejde så mange sammen?}

\todo{status meetings}

\todo{Redmine? Vi bruger det ikke rigtigt}



\chapter{Recommendations for Future Developers}\label{chap:future_dev_recommendations}
\todo{kapitel til alt hvad vi syntes de næste skal vide hvis de ikke har læst resten af rapporten}

\todo{De bør få et ordentligt testmiljø op!}

\section{Process Recommendations}
\todo{Hvordan de bør styre procesen næste år. Anbefalet en samlet, unified backlog!}

\section{Recommandations for Jenkins}
\todo{Hvad der kan gøres for at forbedre jenkins fremover. Referer til den nuværende jenkin struktur appendix}
For future students: How to make new libraries published (server file /srv/jenkins/project\_version\_codes/libversion)
