\chapter{Recommendations for Future Developers}\label{chap:future_dev_recommendations}
As presented in the previous chapter, we have acquired ourselves some experiences from working in a large project setting. To allow the next year's students to learn from these, we make a number of recommendations for future developers that take over the GIRAF project.

\begin{chapterorganization}
  \item in \sectionref{sec:dev_recommentations} we make recommendations regarding the development method;
  \item in \sectionref{sec:ci_recommendations} we recommend improvements for continuous integration.
\end{chapterorganization}

\section{Development Method Recommendations}\label{sec:dev_recommentations}
We recommend that the developers of next year continue using Scrum as their development method. However we suggest that the sub-project structure should be changed such that the sub-projects are divided by functionality and not responsibility (see \chapterref{sec:process_improvements}). We also highly recommend using a common product backlog. We suggest finding a tool that eases the management of the product backlog, such that the backlog continue to be unified and is easy to navigate.

We also recommend that there be arranged some social gatherings, outside of the project. In our experience the gatherings are more successful if they are lightweight, otherwise they need to be planned in greater detail and it is hard to find someone who wants to be responsible for the planning. Examples of gatherings could be eating lunch together every Friday, or agreeing to attend FooBar on a specific day. It is easier to work together if you know the people you are working with.

We suggest that there continue to be roles distributed among groups. It is very important that groups which fulfill roles that have a large amount of interaction with other groups take that responsibility seriously and are available in person during the day. Otherwise that can be a source of frustration for the other groups. Examples of roles with many interactions are Git, Product Owner and Server roles.

\section{Recommendations for Continuous Integration}\label{sec:ci_recommendations}
We have made many improvements to continuous integration and the continuous integration platform Jenkins. There are, however, still improvements to be made:

\begin{description}
  \item[Automatic Deployment of Scripts] Currently most scripts used on the server are not automatically updated. Whenever a change is made to them in the Git repository, they have to be manually updated on the server. This is time consuming and error prone. They should instead be automatically deployed.
  \item[Automatic Monkey Test Failure Notifications] When a monkey test fails, the responsible developers are not automatically notified. Instead they have to manually check the builds or manually subscribe to an RSS feed to get notifications. Monkey tests should be updated so that when a failure occurs the responsible developers are automatically notified, as is the case for other jobs.
  \item[Prevent Push of Snapshots] We have implemented a pre-receive Git hook that prevents developers from pushing dynamics dependency versions and local snapshots of libraries, because these can be error prone. We have not prevented the pushing of snapshot versions. Snapshots are the newest version of a library in-between releases and can contain errors. Since we only store the most recent snapshot, using a snapshot version can result in a library version being updated erroneous. Therefore references to snapshot versions should not be pushed to the remote.
  \item[Testing] There are virtually no tests for the various GIRAF apps and libraries. While we have made improvements to test automation, the code for the apps and libraries is written in such a way that it is very difficult to test. The code should be updated to make testing easy for developers.
\end{description}

These items can be found in the backlog.