%!TEX root = ../report.tex
\chapter{Sprint Review}\label{chap:sprint4_end}
We have accomplished all the selected tasks in this sprint. The report tasks which were postponed are related to finishing the parts of the report which are not directly related to the work we did this sprint. As written, these report tasks will be completed in the week between the end of the sprint and the project deadline.

\begin{chapterorganization}
    \item in \sectionref{sec:s4_goals} we evaluate how the sprint went and wether we reached our goals on a group level.
    \item in \sectionref{sec:s4_multiprj_review} we evaluate how the sprint went on the multi-project level.
\end{chapterorganization}

\section{Sprint Goals}\label{sec:s4_goals}
\begin{description}
    \item[Prioritizing Libraries (1)] We have changed the build queue in Jenkins to be a prioritized queue, which uses scheduling that favors libraries over apps, but still prevents starvation.
    \item[Run monkey tests on debug builds (2)] We have changed our monkey test scripts such that the tests now are performed on debug builds of the apps instead of release builds.
    \item[Easy download and install (3)] We have made a Powershell and a Bash script which download the newest build of every app from the FTP server and install them on the a connected tablet in an automated manner without additional software on one's device.
    \item[Document Jenkins configuration (4)] We have created a document which details how the jobs are configured as well as which files are important, what they are used for, and where they are. Additionally, the document contains details about the authentication system and recommendations to that. We believe this guide is useful for the future developers.
    \item[Screenshot of monkey test (5)] We have a user story which required some other changes to how the monkey tests are run. When we made those changes it did not require that much effort to also make the monkey test capture screenshots. Now a screenshot is saved from each device after a monkey test has finished, including if it crashed.
    \item[Decrease build times in Jenkins (6)] We have improved the build times in Jenkins. Now we have a pool of tablets available for testing, such that we do not have to start a emulator. This saves time every build. We have decreased the build time by approximately \SI{30}{\percent}.
\end{description}
\section{Multi-Project Sprint Review}\label{sec:s4_multiprj_review}
The sprint review is held on May 20. There was some stress leading up to the sprint review because of server troubles. We had a server breakdown on May 14 and it was restored on May 18 to the state it was in the morning of May 13. The server was down two work days, which resulted in reduced productivity among all groups. There was no loss of code, because of the decentralized way Git works --- the developers still had the code locally.  However, the changes made to the configuration of Jenkins during that period was lost. We did not make many changes, so it was quickly restored once the server was up again.

\subsection{Analysis of Server Breakdown}
We analyze the causes of the server breakdown to avoid a similar situation in the future. According to the server responsible group, the problem started when the server has no disk space left. This caused all Git pushes to be rejected and every Jenkins build to fail. The server was granted more space, which was to be added to a specific partition. However, they accidentially corrupted the partition group which caused the server to be unable to mount it. They restored the server to a two-days-old state.

The main lesson to be learned from this experience is that technical problems should be expected. Because we do not have a professional server administrator, we cannot expect configuration changes always to be implemented successfully. We should accommodate this kind of problems by resolving problems in their early stages. We recommend that the server responsible group keep track of the server status and for example set an email notification when the available disk space is low. Ideally the technical operation should be performed at times where developers are not working, for example in weekends. This also allows time to restore the server if something goes wrong.

Because we lost configuration changes in Jenkins, we also recommend to create separate backups of the configuration to somewhere other than the server to prevent such an issue in the future. Other than the Jenkins configuration, nothing was lost during the breakdown.
