\chapter{Improving Monkey Testing}
We make some changes to the way monkey tests are performed this sprint.

\begin{chapterorganization}
  \item in \sectionref{sec:monkey_test_debug_app} we make monkey tests run on debug apps rather than release apps;
  \item in \sectionref{sec:monkey_in_parallel} we update monkey tests to be able to run in parallel on all connected devices, because of the addition of physical devices being used by Jenkins now;
  \item in \sectionref{sec:monkey_test_flow} we give a summary of the new monkey test flow.
\end{chapterorganization}

\section{Running Monkey Tests on Debug Apps}\label{sec:monkey_test_debug_app}
At the moment, monkey tests are run on release versions of the apps. This version uses the production database, which, until now, works fine, because local changes are not persisted on the database. However, this is to change as the \db{} groups work on automatic synchronization between devices and the production database. Obviously, we do not want the monkey tests to insert dummy data into the production database. They could accidentally delete or modify customer data if we allow it to test on the production database. We therefore need to make the monkey test the debug version of the apps instead, as the debug versions automatically use a development database. This corresponds to user story 3.

Prior to running a monkey test, the application to test has to be installed on the device. To make this installation easy, we put the newest version of each compiled debug application on the FTP server using the bash script shown in \listingref{lst:move_debug_apk}. The overall purpose of the script is to first delete the old version of the application from the FTP directory and then move the new one into the FTP directory. The application file is named using the scheme \code{[package name]\_v[version name]b[build number]\_debug\_aligned.apk}. To know which file corresponds to a specific application, we need the package name of the application. Part of this is found in line \ref{move_d:package} by performing the following steps:

\begin{enumerate}
  \item The package name is contained in the start of the name of the APK file. We use the \code{find} command to search for file names in the project directory which end in \mono{\_debug\_aligned.apk}.
  \item Afterwards, we pipe the file names to the \code{grep} command and use a regular expression to match anything up to and including the package name. The \code{P} option specifies the Perl regular expression syntax. The \code{o} options makes it so that only the matching part of the input string is printed. For example, \mono{grep} with the input \code{applications/dk.aau.cs.giraf.launcher\_v2.4b2\_debug\_aligned.apk} prints \code{applications/dk.aau.cs.giraf.launcher} (notice that the last part is the package name).
  \item The result of the match is stored in the variable \code{PACKAGE\_W\_PATH}
\end{enumerate}

In line \ref{move_d:empty-check} we check if an APK matching the naming scheme was found, by checking if \code{PACKAGE\_W\_PATH} is empty (the \code{z} option). If the resulting match is empty we exit, otherwise, we remove the path from the variable \code{PACKAGE\_W\_PATH} so we have only the package name left. We use this package name to remove the old apk from the FTP folder before we move the new APK to the FTP folder (lines \ref{move_d:move_start}--\ref{move_d:move_end}).

\begin{lstlisting}[language=bash,showstringspaces=false,caption=Script that moves the debug APK to the ftp server,label=lst:move_debug_apk]
#!/bin/bash

FTP_DIR="/srv/ftp/debug_apks/"
PACKAGE_W_PATH=$(find . -type f -name "*_debug_aligned.apk" -print | grep ".+(?=_v.+b[0-9]+_debug_aligned\.apk)" -Po) (*@\label{move_d:package}@*)
echo "FTP dir: "$FTP_DIR

if [ -z "$PACKAGE_W_PATH" ] (*@\label{move_d:empty-check}@*)
then
    echo "No file found"
    exit 1
else
    PACKAGE=$(basename $PACKAGE_W_PATH); (*@\label{move_d:move_start}@*)
    echo "Package: $PACKAGE"
    echo "Remove old files: $FTP_DIR$PACKAGE"*.apk
    rm "$FTP_DIR$PACKAGE"*.apk
    find . -type f -name "*_debug_aligned.apk" -print -exec mv {} "$FTP_DIR" \;
    exit 0 (*@\label{move_d:move_end}@*)
fi
\end{lstlisting}

We add the execution of the move script as a post-build task on all app jobs on Jenkins. We also update the monkey jobs to use the debug versions instead of the release versions. This means that when new apps are built, the debug versions are moved to the debug folder in the FTP directory. Every night, the monkey tests use these debug versions for testing. Hence, the user story is solved.

\section{Running Monkey Tests in Parallel}\label{sec:monkey_in_parallel}
With the addition of being able to run tests on multiple physical devices, we need to update monkey tests so that they run in parallel on all connected devices. To do this, we first update our script that installs APKs on a device so that it does so in parallel on all connected devices (see \appendixref{lst:uninstall_apks}). We do the same for the script that runs the dummy database insertion app (see \appendixref{app:start_wait_db_inserter_parallel}). The modifications to these two scripts are similar to that of the the uninstallation script (\listingref{lst:uninstall_apks}). Finally to run the monkey tests in parallel we make a new script, as the Jenkins emulator plugin does not support this feature. The script can be seen in \listingref{lst:run_monkey_parallel}. We choose not to extend the Jenkins plugin with this feature, as it is faster for us to make a script.

Before starting the script we send a broadcast to the Simiasque app that blocks the notification bar on line \ref{run_monkey_parallel:go_away_monkey_true}. After running the test we unblock the notifaction bar (line \ref{run_monkey_parallel:go_away_monkey_false}).

On line \ref{run_monkey_parallel:run_monkey_test}, in the \code{run\_monkey} function, we run the monkey test on the device specified as the first argument (\code{\$1}). We place the output in a temporary log file. In line \ref{run_monkey_parallel:screenshot} we take a screenshot of the device using the ADB tool \parencite{stackoverflow-adb-screencap2014}. Thus, there will be a screenshot of the device just as it finished running the monkey test, or as the monkey test crashed the app. \todo{Vi færdiggør her en user story}

Since many devices may be connected, it is important to know on which device a test was performed. To record this information, we get relevant device information in line \ref{run_monkey_parallel:device_info}. The \code{getprop} contains many properties of the device. We select the device information and its SDK version \code{grep -E "product|sdk|serial"}. Using the \code{E} flag enables extended regular expressions allowing us to use \emph{or} (pipe). We also remove any information matching \code{ro.boot}, as this is not relevant. Finally the device information is concatenated with the temporary log file into a log file.

Having run the monkey test and stored log files for all devices in line \ref{run_monkey_parallel:parallel_loop} we must check if any of the tests failed. If a test succeeded, the final line of the log file will contain \code{monkey finished}. If it failed, it will state that it appears to have crashed. Thus we simply check the last line of the log file if it contains \code{monkey finished}, and if not report a failure. We do this in lines \ref{run_monkey_parallel:check_output}--\ref{run_monkey_parallel:monkey_failed}. We check all log files if they report failure and exit with \code{1} if any of the log files report failure (line \ref{run_monkey_parallel:exit_1}).

\begin{lstlisting}[language=bash,caption=Script that runs monkey tests on all connected devices in parallel,label=lst:run_monkey_parallel]
#!/bin/bash
BASE_OUTPUT_NAME=${JOB_NAME}_b${BUILD_NUMBER}

run_monkey() {
    PACKAGE=$2
    BASE_OUTPUT_NAME=$3
    SEED=1
    NO_EVENTS=1000
    DELAY=10 # in ms

    $ANDROID_HOME/platform-tools/adb -s $1 shell am broadcast -a org.thisisafactory.simiasque.SET_OVERLAY --ez enable true (*@\label{run_monkey_parallel:go_away_monkey_true}@*)

    $ANDROID_HOME/platform-tools/adb -s $1 shell monkey --kill-process-after-error -v -v -s $SEED --throttle $DELAY -p $PACKAGE $NO_EVENTS > ${BASE_OUTPUT_NAME}_$1.log.tmp (*@\label{run_monkey_parallel:run_monkey_test}@*)
    $ANDROID_HOME/platform-tools/adb -s $1 shell screencap -p | sed 's/\r$//' > ${BASE_OUTPUT_NAME}_$1.png (*@\label{run_monkey_parallel:screenshot}@*)

    $ANDROID_HOME/platform-tools/adb -s $1 shell am broadcast -a org.thisisafactory.simiasque.SET_OVERLAY --ez enable false (*@\label{run_monkey_parallel:go_away_monkey_false}@*)

    # Get device information
    $ANDROID_HOME/platform-tools/adb -s $1 shell getprop | grep -E "product|sdk|serial" | grep -v ro.boot | cat - ${BASE_OUTPUT_NAME}_$1.log.tmp > ${BASE_OUTPUT_NAME}_$1.log (*@\label{run_monkey_parallel:device_info}@*)

    rm ${BASE_OUTPUT_NAME}_$1.log.tmp
}
export -f run_monkey

SERIAL_NUMBER="$(/srv/scripts/get_serial_numbers.sh)"

if [ "$SERIAL_NUMBER" ]
then
    parallel run_monkey {1} $1 $BASE_OUTPUT_NAME ::: $SERIAL_NUMBER (*@\label{run_monkey_parallel:parallel_loop}@*)

    MONKEY_FAILED=false
    for s in $SERIAL_NUMBER
    do
        MONKEY_FINISHED="$(tail -1 ${BASE_OUTPUT_NAME}_${s}.log | grep -i "monkey finished")" (*@\label{run_monkey_parallel:check_output}@*)

        if ! [ "$MONKEY_FINISHED" ]
        then
            echo "Monkey test on $s appears to have crashed"
            MONKEY_FAILED=true (*@\label{run_monkey_parallel:monkey_failed}@*)
        fi
    done

    if [ "$MONKEY_FAILED" = true ]
    then
        exit 1 (*@\label{run_monkey_parallel:exit_1}@*)
    fi
else
    echo "No device found"
    exit 1
fi
\end{lstlisting}

\section{Preventing Monkey Test from Disabling WiFi}
\dummy

\section{Monkey Test Flow}\label{sec:monkey_test_flow}
The setup of monkey testing has been through several changes since its inception. The current flow of a monkey test is as follows:

\begin{description}
  \item[1) Connect to Devices/Run Emulator] We start by trying to connect to any physical devices. If there are no devices to connect to, we start an emulator instead.
  \item[2) Install APKs] The APKs needed for the monkey test are installed. This mostly includes the Launcher app and Pictosearch app, but varies from app to app. The dummy database inserter app and the app to be tested are also installed. The APKs are installed in parallel on all connected devices.
  \item[3) Run Dummy Database Inserter] The dummy database inserter app is run in parallel on all connected devices after having installed all APKs. We do this so that apps do not have to download the full database which can take 10 minutes or more. The build waits for the dummy database inserter app to finish inserting data before continuing.
  \item[4) Enable Simiasque] We enable Simiasque to prevent the monkey test from disabling Wifi.
  \item[5) Run Monkey Test and Report Any Failures] The monkey test is run in parallel on all connected devices. A log file is produced for each device containing relevant device information. A screenshot is also made immediately after the app has either finished monkey testing or crashed. If a failure occured on any device, a failure is reported. 
  \item[6) Disable Simiasque] Simiasque is disabled so that the notification bar can be used again.
  \item[7) Publish Log Files and Screenshots] The log files and screenshots are published on Jenkins so that they can be accessed easily.
\end{description}