\chapter{Pre-Receive Hook}
During the latest sprints, we have discovered several times that developers have committed the database password on Git which causes everyone to change the database password. This is frustrating and time consuming, and for this reason we choose to implement a pre-receive Git hook which rejects a push containing the database password. This is not a user story but a refactoring of the version control system. A pre-receive hook is run every time something is pushed to the Git repository, but before the changes are applied (in contrast to the post-receive hook previously described). If the script returns a non-zero exit code, the commit is rejected.

As we develop this hook, we add a number of additional features. Overall, it checks for the following:
\begin{description}
  \item[Database Passwords] The database password is stored in a Java file called \mono{DatabaseCredentials.java}. Every database developer has this file locally and inserts it manually during development. On Jenkins, this file is automatically added to the project during build. Even though the file is part of the gitignore list, it is sometimes commited by mistake. If this file is contained anywhere in the push, it is rejected.
  \item[References to Local Dependencies] Sometimes, applications reference a \code{0.0-SNAPSHOT} version of a library, which is used to reference a library which is stored on the developer's computer. When Jenkins builds the project, it breaks because it cannot find the library. Such mistakes are avoided by checking for local dependency references.
  \item[Dynamic Dependency References] Dependencies can be declared with a dynamic version number (e.g.\@ \code{1.4.+}). As described earlier, we discourage this as it removes the ability to backtrack exact dependency versions in a specific revision of a repository. We therefore check for such references as well and rejects the push of they are referenced.
  \item[Push Enabled] The script checks the contents of a specific file on the server. If the file contains \mono{1}, we instantly reject the push. This allows us to temporarily disable pushes to Git if we for example are restarting Jenkins.
\end{description}
The full pre-receive Git hook code can be seen in \appendixref{sec:app_git_hook_pre}.
