%!TEX root = ../report.tex
\chapter{Sprint Planning}\label{chap:s4_sprintplanning}
This sprint planning proceed ordinarily as specified by the development method. We first have a sprint planning in the subproject followed by sprint planning in the group.

\begin{chapterorganization}
  \item in \sectionref{sec:S4_bd} we describe the user stories we commit ourselves to during the \bd sprint planning meeting.
  \item in \sectionref{sec:S4_group} we describe the sprint planning in our group and lists our tasks with reference to a user story.
\end{chapterorganization}

\section{\bdtitle Sprint Planning}\label{sec:S4_bd}
During the \bd sprint planning we choose to work on four user stories and one technical work item. They are labeled with a number in parentheses for reference. In the last sprint we specified precisely the backlog items used and how to formulate user stories (see \sectionref{sec:processspecs}).

Each user story now has a number of \emph{conditions of satisfaction}. These conditions of satisfaction are listed as bullets below each user story. The conditions have been elicited by asking our customers and our product owner.

\begin{description}
  \item[(1)] As a developer I want libraries to have higher priority than other jobs in the build scheduling on Jenkins so that libraries are not bottlenecked when several people are reliant on them.
  \begin{itemize}
    \item Jobs must not be starved
    \item Libraries must be prioritized higher than other job types
  \end{itemize}
  \item[(2)] As a developer I want tests to run on the debug version of apps so that they can be tested on a test database.
  \begin{itemize}
    \item Monkey tests must run on debug APKs
  \end{itemize}
  \item[(4)] As a developer I want an easy way to download and install all apps so that it is easy to test the apps combined, and easy to show the external customers.
  \begin{itemize}
    \item Users should not have to install additional software onto their computers
    \item Users should be able to merely run a program that then automatically downloads and installs all apps on a device
    \item After running the program, the device must contain the newest version of all apps
  \end{itemize}
  \item[(5)] As a future developer I want to know what is on Jenkins and how it is structured so that it it is easy to start working with
  \begin{itemize}
    \item Document how to configure projects
    \item Document configuration of authentication
    \item Document files used for automation
  \end{itemize}
\end{description}

The following is the technical work item we work on in sprint 3. This technical work item is not formulated as a user story, as it is hard to have conditions of satisfaction for it. The build time can almost always be decreased, and so it would be a never-ending user story. We choose to work on it in this sprint, as it was prioritized highly by the other groups, as there are still improvements to be made, mostly related to the emulator.

\begin{description}
  \item[Decrease Job Build Times on Jenkins (3)] When the queue on Jenkins is long, jobs can take a very long time to build. As we discussed in \sectionref{sec:non-emulator_testing}, the build times can be further reduces by working on the emulator.
\end{description}

\section{Group Sprint Planning}\label{sec:S4_group}
At our internal sprint planning we divide the chosen user stories into tasks and estimate them. For this sprint, we have a total of 70 half days of work. \tableref{tab:sprint4_tasks} shows the tasks we have committed to solve for this sprint. Tasks with a plus (+) are tasks that have been added during the sprint as they were discovered.

In addition to the tasks related to solving our chosen backlog items, we also work on report related tasks for finish our report, such as introduction, conclusion, etc.

\begin{table}%
  \centering
  \begin{tabular}{p{0.6\textwidth}rr}
    \toprule
    \textbf{Task} & \textbf{User Story} & \textbf{Estimation} \\
    \midrule
    Investigate conditions of satisfaction for our user stories & n/a & 2 \\
    Make a program to download and install the newest version of each app & 4 & 2 \\
    Make it so that the Android emulator is only started when no other devices are connected & 3 & 6 \\
    Connect to all Android devices wirelessly & 3 & 2 \\
    Connect Android devices permanently to the Internet (renamet til: set up router) & 3 & 2 \\
    Put debug APKs onto the ftp & 2 & 1 \\
    Make monkey tests use debug APKs & 2 & 1 \\
    Install Jenkins Prioritization Plugin & 1 & 1 \\
    Choose schedule for prioritization plugin & 1 & 2 \\
    Identify areas for Jenkins documentation (spike) & 5 & 2 \\
    Write about Jenkins files on server (from spike) & 5 & 2 \\
    Write about Jenkins structure (from spike) & 5 & 2 \\
    Investigate method to decrease emulator usage time (spike) & 3 & 1 \\
    % \midrule
    % \textbf{Down-prioritized tasks} & & \\
    % \midrule
    % \midrule
    % \textbf{Missed tasks} & & \\
    % \midrule
    % \textbf{Rejected tasks} & & \\
    % \midrule
    \midrule
    \textbf{Original total} & & y \\
    \textbf{Total} & & x \\
    \bottomrule
  \end{tabular}
\caption[Sprint 4 backlog]{Sprint backlog for sprint 4, excluding report tasks. The tasks are listed in no particular order.}
\label{tab:sprint4_tasks}
\end{table}