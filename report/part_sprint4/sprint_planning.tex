\chapter{Sprint Planning}\label{chap:s4_sprintplanning}

\section{\bdtitle Sprint Planning}\label{sec:S4_bd}
During the \bd sprint planning we choose to work on the following user stories in this sprint. They are labeled with a number in parentheses for reference.

\todo{Tilføj conditions of satisfaction}

\begin{description}
  \item[Job Prioritization on Jenkins (1)] Developers have specified a desire for some jobs to have a higher priority than other jobs. For example developers would like libraries to run before apps.
  \item[Monkey Test on Debug Versions of Apps (2)] With the \db groups working on implementing database sync, they are making a test database available. Monkey tests currently use a development database that might be wiped accidentally. To ensure this does not happen monkey tests should run on debug APKs rather than release APKs.
  \item[Decrease Build Times on Jenkins Further (3)] When the queue on Jenkins is long, jobs can take a very long time to build. As we discussed in \sectionref{sec:non-emulator_testing}, the build times can be further reduces by working on the emulator.
  \item[Easy Download and Installation of all Apps (4)] In the previous sprint developers were not always using the latest versions of all apps, which lead to issues at the sprint end. The developers therefore want an easy way to download and install the newest version of all apps on their devices.
  \item[Document Jenkins Structure for Next Semester (5)] To make it easy for the next semester to start working with Jenkins, documentation of the Jenkins structure used is needed.
\end{description}

\section{Group Sprint Planning}\label{sec:S4_group}