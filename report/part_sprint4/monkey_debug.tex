\chapter{Monkey Debug}
At the moment, the monkey tests are run on release versions of the apps. This is problematic as the \db groups are working on switching the release version a live production database. Obviously we do not want the monkey tests to insert dummy data into the production database. It could even accidentally delete or modify some customers data if we allow it to test on the production database. We therefore need to make the monkey test the debug version of the apps instead, as the debug versions should automatically use a development database. This has not been a problem previously as the tablets have not been syncing their local changes up to the central server, but that is about to change.

To make installation easy we put the compiled debug apps on the FTP server. We use a \code{bash} script to move the APK files. \listingref{lst:move_debug_apk} contains the script. Line \ref{move_d:package} finds the package name of any compiled debug APKs by the following steps:

\begin{itemize}
  \item The package name is contained in the start of the name of the APK file. We use the \code{find} command to search for file names which end in ``\_debug\_aligned.apk''.
  \item We then pipe the result to \code{grep} and use a regular expression to match anything up to and including the package name. The \code{P} options specifies the regular expression syntax. The \code{o} options makes it so that only the match is shown.
  \item The result of the match is stored in the variable \code{PACKAGE\_W\_PATH}
\end{itemize}

In line \ref{move_d:empty-check} we check if an APK matching the naming scheme was found, by checking if \code{PACKAGE\_W\_PATH} is empty (the \code{z} option). If the resulting match is empty we exit, otherwise, we remove the path from the variable \code{PACKAGE\_W\_PATH} and use the package name to remove the old apk from the FTP folder before we move the new APK to the FTP folder (line \ref{move_d:move_start}--\ref{move_d:move_end}). We need the package name of the app to remove the old build of the debug version, as we do not want to store more than one debug build of each app.

\begin{lstlisting}[language=bash,showstringspaces=false,caption=Script that moves the debug APK to the ftp server,label=lst:move_debug_apk]
#!/bin/bash

FTP_DIR="/srv/ftp/debug_apks/"
PACKAGE_W_PATH=$(find . -type f -name "*_debug_aligned.apk" -print | grep ".+(?=_v[0-9]+\.[0-9]+b[0-9]+_debug_aligned\.apk)" -Po) (*@\label{move_d:package}@*)
echo "FTP dir: "$FTP_DIR

if [ -z "$PACKAGE_W_PATH" ]; (*@\label{move_d:empty-check}@*)
then
    echo "No file found"
    echo "Package with path: "$PACKAGE_W_PATH
    exit 1
else
    PACKAGE=$(basename $PACKAGE_W_PATH); (*@\label{move_d:move_start}@*)
    echo "Package: $PACKAGE";
    echo "Remove old files: $FTP_DIR$PACKAGE"*.apk;
    rm "$FTP_DIR$PACKAGE"*.apk;
    find . -type f -name "*_debug_aligned.apk" -print -exec mv {} "$FTP_DIR" \;
    exit 0 (*@\label{move_d:move_end}@*)
fi
\end{lstlisting}

We then add the execution of the script as a post build task on all app jobs on Jenkins. 
\todo{Explain what needed to be changed on the monkey jenins job.}
Now the monkey tests are run on debug versions instead of release versions.