\chapter{Further Improving Build Times}
This chapter describes our work on the user story \us{Decrease Job Build Times on Jenkins}. The developers want more rapid response to builds failing or succeeding. As explained in \sectionref{sec:faster_build_evaluation}, the emulator start-up time is a significant part of the build time, taking approximately two minutes. The emulator is used for running the tests, and as such we have to find an alternative method of running the tests.

\begin{chapterorganization}
  \item test
\end{chapterorganization}

\section{Selecting a Non-Emulator Test Method}
As explained in \sectionref{sec:non-emulator_testing} there are two ways of running tests without an emulator: testing on a physical device, and testing on the Java Virtual Machine. Testing on a physical device is the only way to realistically run tests. All groups have a tablet laying unused much of the day as well as all night, and we will utilize these unused tablets. However, it will be too tedious connecting these tablets physically to the server (the server is behind locks in the basement of Aalborg University) and we most likely will not get permission to physically connect USB devices to it.

\section{The Overall Idea}
We want to create a pool of available tablets, such that they in an easy way can be part of testing. Our plan is to setup a wireless router in vicinity of our group rooms. The wireless router broadcasts a Wi-Fi signal, and any tablet connected to this Wi-Fi will automatically be in a pool of connected tablets, available for testing. The server is able to communicate with the wireless router through the network at Aalborg University. We have acquired a wireless router from the semester coordinator, Ulrik Nyman \todo{sp}, a \emph{TP-Link TL-WDR3600}.

\section{Creating a Pool of Test Tablets}
In order to transform a garden variety wireless router into a device that maintains a pool of tablets, we flash it custom firmware. We have chosen to flash with \emph{OpenWRT}, an open source router firmware \todo{Insert citation}. OpenWRT is chosen because of earlier experience in this group with this firmware, as well as being compatible with our device and having all features we need. After flashing, we setup the basic system settings and configure the Wi-Fi network as one normally will in any home router.

We furthermore setup port forwarding to the tablets. Any device connecting to the Wi-Fi will \todo{automatically get a static IP and a port forward.}

Script is called in a cron job. Because stability and if it crashes it will be executes again. Minimum resolution is 1 minute though, which is adequate. \todo{temp.}

\section{Creating Route from Server to Tablet}
Testing the above setup revealed that the server and our router is separated on the network. The only traffic we can get through is ping. We went to IT Services at Aalborg University and got our router moved to the same network as the server. \todo{Skriv dette pænere og mere fyldestgørende?}