\chapter{Upload of Apps to Google Play}\label{sec:upload_google_play}
Jenkins compiles and builds the Giraf projects but does nothing with the generated APKs. One of the user stories selected in this sprint is \us{automatic upload of alpha releases to Google Play}.

Before the APKs can be published to Google Play, they need to be signed with a signature. The Android plugin for Gradle has functionality for automatic signing of APKs, and we will use this to sign. A keystore file is used to sign, and we are not interested in everybody having this file as it serves as a proof of identification. We save the file on the server, which means that it becomes impossible to build release versions of the apps locally using the Giraf keystore. When uploading a new app to Google Play, the version code of the new app must be greater than the version code of the app already in the app store. Incrementing the version code is not done per default, so we need to set up the build to increase the version code every time an app is successfully built. 

We have written a Gradle plugin handling this with the major part of this seen in \listingref{lst:gradle_versioncode}. The plugin keeps a properties file with the current version code for all apps. Upon executing the \mono{increaseVersionCode} Gradle task, it reads the version code (line 8) from the properties file and increments it (line 12). Then it writes it to the app manifest file (line 13), such that the build following will use the new version code. Finally, it updates the property file with the new version code (line 15), such that the next build will use this version code. If no current version code is found in the properties file (line 21), we assume the app is new and starts at version code \mono{1}, and as such we do not increment it (line 10).
\begin{gradlecode}[caption=Part of our Gradle plugin for updating version code,label=lst:gradle_versioncode]
project.task('increaseVersionCode') << {
    // [...]
    // Check to see if properties file exists.
    if (project.file(versionCodesFilePath).exists() != true) {
        throw new GradleException("No version code file found. Only Jenkins should run this task")
    }
    // [...]
    def newVersion = getVersionCode(project, applicationId)
    def versionCode = newVersion['value']
    if (!newVersion['created']) {
        // Increment version code if not new
        versionCode++
        // Write incremented version code to manifest
        // [...]
        // Write incremented version code back to properties file
        // [...]
    }
}

def getVersionCode(project, applicationId) {
    // Reads the version code from the properties file. Returns 1 if the version code does not exist.
}
\end{gradlecode}
 
Now that the APKs have been signed, they need to be uploaded to Google Play. This can be done in two ways: Via a Jenkins plugin \parencite{jenkins-play-plugin} or through Gradle \parencite{gradle-play-plugin}. The Jenkins plugin is easy to use, but requires that the exact name and location of the of the APK to upload is known. The Gradle plugin knows this already, since the information is already present in the Gradle build environment. Therefore we use Gradle to publish to Google Play via a Google Play API\@. After the APKs have been uploaded, they are moved to the ftp server which is hosted on the same machine as Jenkins in the directory \mono{/srv/ftp/}. This way the APKs are also available outside of the Google Play store.