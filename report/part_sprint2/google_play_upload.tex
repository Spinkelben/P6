\chapter{Upload of Apps to Google Play}\label{sec:upload_google_play}
One of the user stories selected in this sprint is ``Automatic upload of alpha releases to google play''. Currently the Jenkins server compiles the code but does nothing with the APKs. Before the APKs can be publish on Google Play, they need to be signed with our signature. The android plug-in for Gradle has functionality for automatic signing of APKs. We use this to sign. To sign we need a keystore file. We are not interested in everybody having this file as it serves as a proof of identification. We save the file on the server, which means that it becomes impossible to build a release version of the apps locally.
Every app has a version code, which is incremented every time a new release is published. This is not done per default, so we need to set up the build to increase the version code every time a app is successfully built. 
Now the apks have been signed, we need to upload them. This can be done in two ways: via Jenkins \parencite{jenkins-play-plugin} or through Gradle \parencite{gradle-play-plugin}. The Jenkins plugin is easy to use, but requires that you know the exact name and location of the of the APK to upload.
\todo{Skriv om plugin her}