\chapter{Upload of Apps to Google Play}\label{sec:upload_google_play}
Jenkins compiles and builds the projects but does nothing with the generated APKs. One of the user stories selected in this sprint is \us{automatic upload of alpha releases to Google Play}.

Before the APKs can be published to Google Play, they need to be signed with our signature. The Android plug-in for Gradle has functionality for automatic signing of APKs, and we will use this to sign. A keystore file is used to sign, and we are not interested in everybody having this file as it serves as a proof of identification. We save the file on the server, which means that it becomes impossible to build release versions of the apps locally.
Google Play store requires an incrementing build number of every app. Incrementing the build number is not done per default, so we need to set up the build to increase the version code every time a app is successfully built. \todo{Write about how increase version code.} \todo{Er det build number eller version code?}
 
Now the apks have been signed, they need to be uploaded. This can be done in two ways: via a Jenkins plugin \parencite{jenkins-play-plugin} or through Gradle \parencite{gradle-play-plugin}. The Jenkins plugin is easy to use, but requires that you know the exact name and location of the of the APK to upload.
\todo{Skriv om plugin her}

\todo{Skriv om vores endelige løsning.}