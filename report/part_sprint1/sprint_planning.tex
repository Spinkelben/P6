%!TEX root = ../report.tex
\chapter{Sprint Planning}\label{chap:sprint1_planning}
The focus of this sprint is to make the code base from the previous year work without major bugs etc., as pointed out by the semester coordinator. As mentioned in \sectionref{sec:meetings}, there is a separate sprint planning meeting for each subproject. However, the first sprint is different, because the process is not fully specified at this point. Instead, sprint planning meetings are held on three levels: Multi-project, subproject, and group. Before that can happen, we need to look at the backlog from last year.

\begin{chapterorganization}
  \item in \sectionref{sec:s1p_userstory} we describe how we aggregate all known user stories from previous years;
  \item in \sectionref{sec:s1p_multiproject} we describe the multi-project planning meeting, especially how user stories are selected;
  \item in \sectionref{sec:s1p_bd} we describe the delegation of user stories at the \bd sprint planning meeting;
  \item in \sectionref{sec:group_sprint_planning} we describe our own group sprint planning;
  \item in \sectionref{sec:s1p_response} we evaluate problems that arose during sprint planning.
\end{chapterorganization}

\section{User Stories Aggregation}\label{sec:s1p_userstory}
The students from last year had no centralized product backlog, despite the fact that the previous year had used Scrum as well. We therefore ask all groups at the start of the multi-project to read a report from the previous year and look for candidate user stories. These are then aggregated in a shared document. The user stories are categorized according to the subproject under which they relate the most to. This is only done during the first sprint, as there will be a centralized backlog for future sprint planning meetings.

\section{Multi-Project Planning Meeting}\label{sec:s1p_multiproject}
With all the reports of last year read and the backlog updated to reflect the status of the multi-project, the sprint planning can commence. All groups are present at this meeting, and some pre-planning game activities like project vision are discussed. The newly created backlog is reviewed and each subproject select the user stories they will work on and adds them to the release backlog for sprint 1. The stories are selected according to importance for the customer, which can be other groups as well as the external customer. The priority of user stories at this time is unclear, since at this time there has been no review of the backlog with the external customers. The priority is therefore the same as was last year, but this may have changed since then.

\section{B\&D Sprint Planning}\label{sec:s1p_bd}
After the multi-project sprint planning meeting, the \bd subproject holds a sprint planning meeting. At this meeting, the \bd groups select user stories for their sprint release backlog from the \bd subproject release backlog. This is done such that each group has influence on what they are working on and commit themselves to getting it done. It is hard to know in advance how much work each story contains without dividing it into tasks. We chose not to divide the stories at this meeting, to keep it as short as possible at the cost of precise time estimates. The groups select stories until they feel they have a sufficient work load. The uncertainty of the stories is problematic, as groups may find that they are overburdened or find themselves out of work too soon. We select a single user story, namely \us{Continuous Build \& Integration}. We commit ourselves to this user story because we think our knowledge about the development method is applicable in order to create an automated build and deployment process that supports the overall development method. We are aware that it is a very vaguely defined story, but there are many uncertainties about the multi-project in this initial sprint which makes it difficult to make more precise user stories.

\section{Group Sprint Planning}\label{sec:group_sprint_planning}
After the \bd sprint planning meeting, we meet in the group and do the sprint planning for our group. We split the user story which we chose in the \bd sprint planning meeting and divide it into tasks. The tasks are added to the sprint backlog of our group. Then we estimate the tasks to see if we have a suitable workload for the sprint. In general a group will select more user stories if they estimate that they have too light a workload, or remove some user stories from the release backlog of the group if they are overburdened. In this case we have a large and vague story, so instead we adjusted the tasks included in the story to fit the available time. The tasks are listed in \tableref{tab:sprint1_tasks}, however the task estimation values are lost and are thus not available in the table. All tasks in the table are related to a user story called \us{Continuous Build \& Integration}.

\begin{table}%
  \centering
  \begin{tabular}{p{0.6\textwidth}rr}
    \toprule
    \textbf{Task} \\
    \midrule
    Upgrade Jenkins                                   \\
    Upgrade Jenkins plugins                           \\
    Automated build and test                          \\
    Setup of roles in Jenkins                         \\
    Setup of Javadoc in Jenkins                       \\
    Investigate lint checking                         \\
    Setup automatic lint checker in Jenkins           \\
    Investigate UI testing                            \\
    Investigate monkey testing                        \\
    Investigate Android unit testing                  \\
    Setup Android unit testing                        \\
    Setup Android unit testing in Jenkins             \\
    Investigate concolic testing                      \\
    Enable Jenkins to start an Android emulator       \\
    Jenkins cannot start (+)                          \\
    Write wiki entry on process                       \\
    Write wiki entry on meetings                      \\
    Write wiki entry on backlog                       \\
    Write wiki entry on roles                         \\
    Setup sending email to people breaking builds     \\
    Setup Android publisher plugin for Jenkins        \\
    \midrule
    \textbf{Missed tasks} & & \\
    \midrule
    Setup monkey testing in Jenkins                   \\
    \midrule
    \textbf{Rejected tasks} & & \\
    \midrule
    Setup concolic testing in Jenkins                 \\
    Write wiki entry about concolic testing           \\
    \bottomrule
  \end{tabular}
\caption{Sprint backlog for sprint 1. The tasks are listed in no particular order.}
\label{tab:sprint1_tasks}
\end{table}

\section{Response to Sprint Planning}\label{sec:s1p_response}
When the first sprint planning was made, it was decided that each subproject would choose what to work on, by selecting user stories to put into the release backlog. The developers of last year documented the requirements of the apps in a traditional requirements document. This led to some confusion amongst the groups as it was not clear how these requirements should be converted to user stories for the release backlog and how these stories should be split into tasks for the sprint backlog. This uncertainty was addressed at the following multi project meeting, where the terms were discussed. The product backlog and release backlog was entered into Redmine by each subproject. In the \bd subproject this task was carried out by our group in collaboration with \group{5} (\bd product owner).