\chapter{Sprint Planning}\label{chap:sprint1_planning}
As mentioned in \sectionref{sec:scrum_meetings} there are three sprint planning meetings. The first sprint is different, because the process is not fully specified at this point. Sprint planning meetings are held on three levels instead: Multi-project, subproject and group \todo{How is this different from the first sentence in this paragraph?}. Before that can happen, we need to look at the backlog from last year.

\section{User Stories Aggregation}
The students from last year had no centralized product backlog, despite the fact that the previous year had used Scrum as well. We therefore ask all groups at the start of the multi-project to read a report from the previous year and look for candidate user stories. These are then aggregated in a shared document. The user stories are categorized according to the subproject under which they relate the most to.

\section{Multi-project Planning Meeting}
With all the reports of last year read and the backlog updated to reflect the status of the multi-project, the sprint planning can commence. All groups are present at this meeting. At this meeting some pre-planning game activities like project vision are discussed. The newly created backlog is reviewed and each subproject select the user stories and adds them to the release backlog for sprint 0. The stories are selected according to importance for the customer, that can be other groups as well as the actual customer.

\section{\bd Sprint Planning}
After the multi-project sprint planning meeting the \bd subproject holds a sprint planning meeting. At this meeting, the groups select user stories for their sprint release backlog from the \bd subproject release backlog. This is done such that each group has influence on what they are working on and commit themselves to getting it done. It is hard to know in advance how much work each story contains without dividing it into tasks. We chose not to divide the stories at this meeting, to keep it as short as possible at the cost of precise time estimates. The groups select stories until they feel they have a sufficient work load. The uncertainty of the stories is problematic, as groups may find that they are overburdened or find themselves out of work too soon. We selected the user story ''Automated build and deployment``. We are aware that it is a very vague story, but there are many uncertainties about the multi-project in this initial sprint.

\section{Group Sprint planning}
After the \bd sprint planning meeting, we meet in the group and do the sprint planning for our group. We split the user story which we chose in the \bd sprint planning meeting and divide it into tasks. The tasks are added to the sprint backlog of the group. Then we estimate the tasks to see if we have a suitable workload for the sprint. In general a group will select more user stories if they estimate that they have too light a workload, or remove some user stories from the release backlog of the group if they are overburdened. In this case we had a large but vague story, so instead we adjusted the tasks included in the story to fit the available time. The tasks can be summarized as follows:

\begin{description}
    \item[Build] Automated build and notify relevant persons on failure.
    \item[Documentation] Automated documentation generation from code with comments.
    \item[Concolic Testing] Investigate and setup automated concolic testing.
    \item[Lint] Automatically run linter on code and generate report.
    \item[Unit test] Automated unit testing of build.
    \item[UI test] Automated UI test. Look into monkey testing. 
\end{description}
\todo{This list feels like it needs some rephrasing to make it sound more natural.}

\section{Response to Sprint Planning}
When the first sprint planning was made, it was decided that each subproject would choose what user stories to put into the release backlog, based on the fact that the first sprint focused on bug fixing and making the applications work. There was, however, some confusion amongst the groups as to how to proceed with filling in the product backlog and release backlog on Redmine. At a later meeting this was clarified and the product backlog and release backlog was filled into Redmine by each subproject. In the \bd subproject this task was carried out by our group in collaboration with \group{5}.
