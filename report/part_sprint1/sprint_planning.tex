\chapter{Sprint Planning}\label{chap:sprint1_planning}
As mentioned in \chapterref{chap:sw_dev_method} there are three sprint planning meetings, however the first sprint is different, because the process is not fully specified at this point. Sprint planning meetings are held on three levels instead: Top, subproject and group. Before that can happen, we need to look at the backlog from last year.

\section{User Stories Aggregation}
The students from last year have no centralized product backlog, despite the fact that the previous year had used Scrum as well. We therefore ask all groups at the start of the multi-project to read a report from the previous year and look for candidate user stories. These are then aggregated in a Google Docs document. The user stories are categorized according to the subproject under which they relate the most to.

When the first sprint planning was made, it was decided that each subproject would choose what user stories to put into the release backlog, based on the fact that the first sprint focused on bug fixing and making the applications work. There was, however, some confusion amongst the groups as to how to precede with filling in the product backlog and release backlog on Redmine. At a later meeting this was clarified and the product backlog and release backlog was filled into Redmine, by each subproject. In the build and deployment subproject this task was carried out by our group in collaboration with \group{5}.
