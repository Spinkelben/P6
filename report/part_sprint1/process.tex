\chapter{Software Development Method}%\label{chap:...}
The multi-project consists of 14 groups of approximately 4 people each. All these groups have to collaborate on building the Giraf applications. The groups are organized into three subprojects covering GUI\@, databases, and build and deployment.

To organize these teams we use SCRUM\@ of SCRUMS\@ on the overall level. The project period is divided into four sprints. There are a total of three levels: The group level, subproject level, and the overall project level. While SCRUM\@ of SCRUMS\@ dictates that SCRUM is followed at all levels, we cannot enforce this. We recommend that all groups follow SCRUM\@ in their groups, but each group is free to decide how they organize internally. This poses some challenges that will be described later.

The subproject level follows SCRUM\@ of SCRUMS\@ in the sense that they hold at minimum two SCRUM\@ meetings a month. In regular SCRUM such meetings are held each day, but this is not necessarily the case in SCRUM\@ of SCRUMS\@.

On the overall project level there is a weekly meeting where we discuss the status of each subproject and evaluate the method and roles. This meeting differs from the daily SCRUM\@ meeting. The purpose of the overall level is to ensure that the roles, work products, and meetings are followed as intended.

The following sections describe in details the roles, work products, and meetings that we use.

\section{Work Products}
In SCRUM we use three work products: Product backlog, release backlog, and sprint backlog.

\subsection{Product Backlog}
On the overall project level we use a product backlog that contains all the user stories for the entire multi-project. The purpose of the overall project level is to maintain the product backlog. The user stories in the product backlog are categorized according to each subproject. This enables each subproject to manage their user stories themselves, removing unnecessary management from the overall project level.

\subsection{Release Backlog}
The release backlog contains those user stories that we must finish in the current sprint. These user stories are decided on sprint planning meetings that each subproject hold.

\subsection{Sprint Backlog}
The sprint 

\section{Roles}
Lorem ipsum.d

\section{Meeting Times}
Lorem ipsum dolor sit amet.

\section{Roles: Scrum Masters}
As the process group, we acted like ``scrum masters'' in some instances.

\section{User Stories Aggregation}
Andreas sad og lavede en masse sager sammen med Lukas?

\section{Software Development Method in Our Group}
We use SCRUM\@ in our group...

\section{Setting up Redmine}
\subsection{Collaboration with Groups}
Collaboration with group 3 (Redmine General) and group 10 (issue tracker) on setting up ``sager'' list\\
Collaboration with group 3 on removing gantt chart