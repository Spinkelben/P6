%!TEX root = ../report.tex
\chapter{Software Development Method}\label{chap:sw_dev_method}

One of the great challenges this semester is to coordinate the development process across all groups. This resembles developing a commercial project. 
Short introduction to this chapter. \dummy~\dummy~\dummy~\dummy~\dummy~\dummy

\begin{chapterorganization}
  \item in \sectionref{sec:project_overview} we describe how the multi-project is organized and introduce the hierarchy in the project;
  \item in \sectionref{sec:redmine} we describe how our responsibility for development process influenced the work of other groups and the project organization tool we use;
  \item in \sectionref{sec:responsibilities} we detail the responsibilities delegated to most groups.
  \kimnote{Igen, jeg kan ikke set hvad dette har med sprint 1 af gøre, jeg synes det virker mere generelt.}
\end{chapterorganization}

\section{Project Organization Method}\label{sec:project_overview}
The multi-project consists of 14 groups of approximately four people each. All groups collaborate on building the Giraf applications. The groups are organized into three subprojects: \emph{GUI}, \emph{databases}, and \emph{build and deployment}.

To organize these teams Scrum of Scrums \todo{Vi burde nok ikke kalde det Scrum of Scrums, da alle grupper ikke kører Scrum, og derfor er det egentlig ikke Scrum of Scrums vi kører. Det er nærmere Scrum of ``something''. Generelt burde vi ikke kalde mange af tingene Scrum, da det ikke er Scrum.} is utilized. \kimnote{I kan godt kalde det scrum of scrum, men et eller andet sted skal i forklare at ikke alle grupper bruger scrum og ingen bruger scrum i sin rene form.} The project period is divided into four sprints. There are a total of three levels: The group level, subproject level, and the multi-project level.

\begin{description}
	\item[Group Level] While Scrum of Scrums dictates that Scrum is followed at all levels, this cannot be enforce at the group level. However, it is recommend that all groups follow Scrum in their practice, but each group is free to decide how they organize themselves internally. This poses some challenges that will be described later.
	\item[Subproject Level] This level follows Scrum of Scrums in the sense that they hold at minimum two Scrum meetings a week. In regular Scrum such meetings are held each day, but this is not necessarily the case in Scrum of Scrums.
	\item[Multi-Project Level] A weekly meeting is held where the product backlog is maintained, the status of each subproject is discussed, and the method and roles are evaluated. This meeting differs from the daily Scrum meeting. The purpose of the overall level is to ensure that the roles, work products, and meetings are followed as intended.
\end{description}

\subsection{Scrum Work Products}
From Scrum three work products are used: Product backlog, release backlog, and sprint backlog.

\begin{description}

\item[Product Backlog]
On the multi-project level a product backlog that contains all the user stories for the entire multi-project is used.  
\kimnote{Dette er første gang i snakker om user stories, Jeg synes i bør nævne hvor i får disse fra før i snakker om hvordan de bliver brugt af jeres work products.}
The purpose of the multi-project level is to maintain the product backlog. The user stories in the product backlog are categorized according to each subproject. This enables each subproject to manage their user stories themselves, removing unnecessary management from the multi-project level.

\item[Release Backlog]
The release backlog contains those user stories that must be finished by the end of the current sprint. These user stories are decided on sprint planning meetings that each subproject hold.

\item[Sprint Backlog]
When user stories are selected from the product backlog into the release backlog, those user stories are divided into tasks. These tasks constitute the sprint backlog. It is up to each subproject to decide how these tasks are made. They can do it at the subproject sprint planning meeting or they can decide to delegate the task creation to each group based on their selected user stories.
\kimnote{Det er vigtigt at i forklarer hvordan i gør og hvorfor det giver mening.}

This is where there are difficulties from the fact that not every group use Scrum at the group level. If a group does not use Scrum, there is no guarantee that they will make tasks, or even update tasks should they have been created at the subproject level. If not every group follows Scrum and makes tasks, then it is not possible to measure the progress and productivity of the entire team. In addition if there are no estimated tasks, it is impossible to know at a sprint planning whether or not the multi-project will be able to deliver the release backlog.
\kimnote{Er i sikre på at det er umuligt at vide hvornår man er færdige med sin udvikling hvis man ikke bruger scrum? :P}
\kimnote{Her begynder i på en diskution som er lidt mærkeligt placeret. I har ikke diskuteret produkt eller release backlog, kun sprint backlog. Envidere så nævner i kun problemerne, hvordan løser i disse problemer?}

\end{description}

\subsection{Scrum Roles}
In Scrum there are two roles besides the Scrum team:

\begin{description}
  \item[Scrum Master] On the multi-project level our group is responsible for the development method. This means that our group acts as Scrum master on the multi-project level. On the subproject level, one person has been designated the meeting responsible --- they make sure that the Scrum meetings on this level are held. On the group level each group is free to do what they see fit, so there is no guarantee that there is a Scrum master on this level.
  \kimnote{Hvad med jer, hvad gør i?}
  \item[Product Owner] The product owner (PO) is responsible for maintaining contact with the customers and being their representatives. In the multi-project there are three POs, one for each subproject. The customer for the GUI PO is the end customer, whereas the customers for the database and build and deployment POs are the groups from the other subprojects, as illustrated in \figureref{fig:po_illu}. These roles have been designated to a group in each subproject.
  \kimnote{dette er første gang i nævner "end customers". Beskriv "customers" og "end customer" inden i bruger disse koncepter.}
\end{description}

\fig{po_illu}{Illustration of product owners}{Illustration of product owners. \todo{Temporary figure. Make a nicer in tikz and add some more description. In particular of the arrows.}}

\subsection{Meetings}
Each week there is a meeting at the multi-project level. At this meeting the status of the groups in the multi-project is presented, and the development method is evaluated. This ensures that all groups knows the current status of the project, and that the development method can be continuously \todo{Er det den rigtige cont?} improved.

At the subproject level a Scrum meeting is held at least two times a week. It is up to each subproject to organize the meetings. In addition to the Scrum meeting at the subproject level, each group is advised to hold a Daily Scrum in connection with the advise that they follow Scrum internally, but the it is an advice only.
\kimnote{Scrum of Scrum?}

Since there are three product owners in the entire multi-project, three sprint end and sprint planning meetings are made.

\begin{description}
  \item[Sprint Planning (GUI)] At this sprint planning only one representative from each GUI group participate as pigs (people who are allowed to talk), as well as the GUI PO and one representative from the other POs (although they have a less important role at this meeting). All other people in the multi-project are chicken (people who are not allowed to talk). This meeting is held with the end customers. By making it so that mainly the GUI groups are pigs, the end customers are abstracted from internal development such as continuous integration and database management.
  \kimnote{Dette punkt er godt skrevet fordi i given en forklaring på hvorfor i gør som i gør. Dette er ikke den første gang i skriver godt, dette er bare en godt eksemple :D}
  \item[Sprint Planning (DB and B\&D)] There is a sprint planning for DB and B\&D each, where each subproject establishes the needs of the other subprojects. This meeting is internal, and the end customers do not participate at this meeting, since the DB and B\&D do not have these as customers. DB and B\&D will mainly \kimnote{What do you mean by mainly?} have the GUI groups as their customers, who need various services from the DB and B\&D subprojects, in order for them to fulfil their own user stories directly related to the end customers. Again this abstracts the end customers from internal development.
  \item[Sprint End (GUI)] At this meeting the end customers are presented with the work of the GUI groups. Only the GUI groups participate as pigs at this meeting.
  \item[Sprint End (DB and B\&D)] There are internal meetings in the subprojects and a common meeting. At the common meeting DB and B\&D present the work they have done to the other subprojects. This way the end customers do not have to be bored with databases, build configurations, and such. The reason for having these three POs is that it enables the multi-project to abstract internal development from the end customers, so that they are only concerned with things that have an immediate interest to them.
\end{description}

\subsection{Meeting Times}
A vital part of working agile involves being able to easily communicate with all members of the team. To ensure this it was decided that all groups should be present in their group rooms all weekdays from at least 09:00--15:00. However, as it cannot enforce that each group follows Scrum\todo{We can't enforce anything because we have no actual power over poeple. Scrum or not.}, there is no guarantee that groups will follow this schedule. In fact, two groups specifically said that they cannot. They would, however, be available on e-mail at the same times.


\subsection{Software Development Method in Our Group}
As all groups are recommended to follow the Scrum development method, our group also follows it. \todo{Write that agile/scrum is suited for this product, changing requirements and such.}
\kimnote{Bare ignorer tidligere kommentarer omkring manglende beskrivelse af jeres process. I skal dog forklare i højre grad hvordan jeres process er. Det undre mig atRedmine omtalt før det bliver forklaret. }

In our group we use a \emph{burndown chart} to measure our progress. We would have liked to have this for the entire multi-project as well, but since not every group follows Scrum and makes estimated tasks, it is impossible to measure the performance of the multi-project, and as such it does not make sense to do a burn down chart for the entire multi-project.

We also initially used Redmine as our Scrum board, but due to Redmine being very slow and tedious to update, we replaced it with a physical Scrum board. As not all groups do Scrum and thus estimate tasks, this decision has no impact on the performance of the multi-project.

\section{Setting up Redmine}\label{sec:redmine}
The multi-project uses an open source project management tool called Redmine. The main features used are the wiki, forum, and issue tracker. The issue tracker is not used as a traditional issue tracker, but instead customized to incorporate some Scrum practices. The issue tracker is now able to hold product, release, and sprint backlog. We collaborated with \group{3} (Redmine general) and \group{10} (issue tracker) to set this up. 

Also, a Gantt chart feature had been installed by previous years. The Scrum method does not advocate Gantt charts or other dependency charts. As such, we collaborated with \group{3} to remove this from Redmine.

\section{Other Responsibilities}\label{sec:responsibilities}
\kimnote{"Other resposibilities" som har noget at gøre med hele projekt gruppen bør stå i forlængelse af Project Organization Method.}
In addition to the Scrum roles previously described, the multi-project contains other responsibilities that have been assigned to the groups within the multi-project. These roles include management of Redmine, server, Git, Jenkins and other various tasks. Following they are shortly described:

\begin{description}
  \item[Redmine --- General] \dummy \dummy
  \item[Redmine --- Wiki] \dummy \dummy
  \item[Redmine --- Forum] \dummy
  \item[Redmine --- Issue Tracker] \dummy \dummy
  \item[Server] \dummy \dummy
  \item[Git] \dummy \dummy
  \item[Jenkins] Our group
  \item[Process] Our group
  \item[Code Style] \dummy \dummy
  \item[Customer Relations] \todo{Har vi disbanded/merged med GUI product owner. Skal vi skrive om denne process?}
  \item[Web Administrator] \dummy \dummy
  \item[Android Guru] \dummy \dummy
  \item[Google Analytics and Google Play] \dummy \dummy
  \item[Graphics] \dummy \dummy
\end{description}