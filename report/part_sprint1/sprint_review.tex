\chapter{Sprint Review}\label{chap:sprint1_end}
This sprint had a number of tasks that we wanted to solve and in addition we introduced some changes to the development method. This chapter evaluates the tasks we planned and the process we followed during this sprint.

\section{Sprint Goals}
For this sprint, our main task was to setup continuous integration. The following evaluates the tasks described in \sectionref{sec:group_sprint_planning}.

\begin{description}
  \item[Build] We setup Jenkins to automatically build a project whenever a push is made to the Git repository of that project. The build times, however, are very slow. It can take up to 30 minutes or more to build and test some projects, which is significantly more than the 10 minutes recommended by \textcite{beck2004}.

  For integrating code with the mainline, we use a simple merge strategy which allows developers to commit directly to the master branch. The builds have generally been unstable on Jenkins. This is a potential problem because it can block developers from writing code in projects that do not work. However, because there have not been any complaints from the developers, we assume that it is not a big problem in practice.
  \item[Documentation] Documentation is automatically generated each night from the entire code base and uploaded to a publicly available website. This works as required by the task.
  \item[Concolic Testing] We chose not to implement concolic testing due to it being too complex to implement compared to the benefit it would give us. It may be worth looking more into this in a future sprint if our customers express an interest in this.
  \item[Lint] Lint warnings and errors are automatically generated on Jenkins. We decided not to make a lint error break the build as to not punish groups for lint errors caused by code from previous years. Lint errors are seldom critical, and by displaying the lint errors on the Jenkins page we hope that this will motivate the groups to fix the problems. We find that this does indeed work for some groups.
  \item[Unit Test] We setup unit testing such that they run as part of a build on Jenkins. When a test fails, the build fails. However, testing takes a long time because they run on an emulator which must start up as part of the test process.
  \item[UI Test] We wanted to implement monkey testing, but because the plugin on Jenkins did not support multiple app monkey testing and an unclear structure of APKs on the server, we did not manage to implement monkey testing on Jenkins. However, we have made an improvement to the Jenkins plugin which makes it easier to eventually implement monkey testing in a future sprint.
\end{description}

\section{Process Evaluation During Sprint}
During sprint 1 we changed a few things in the development method of the multi-project:

\begin{description}
  \item[Customer Relations Role] The \emph{customer relations} role was initially vague in its description. Some understood it as being a product owner (PO) role, while others understood it only as a contact person for the external customers. We solved this by letting this role be the contact person for the external customers and assigning the PO role to other groups as explained below.
  \item[Multiple Product Owners and Meetings] Originally, a single sprint end meeting with the external customers was planned. Due to concerns from the semester coordinator, it was changed into several meetings. Last year, there was one sprint end meeting, and it was observed that the external customers were not interested in most of the progress the \db and \bd groups had made. Therefore we decided that the external customers should only attend a smaller meeting where the \gui groups would present their progress as well as the relevant progress from \db and \bd (if any). Expanding on this, we ended up with the structure described in \sectionref{sec:project_overview}, where there are three POs, three sprint end meetings, and three sprint planning meetings.
  \item[Office Hours] In the beginning of the multi-project, it was decided that all groups must be present in their group rooms from 09:00--15:00. However, after the decision had been made, some groups said they not follow this requirement\kimnote{rephrase}. This meant that other groups complained about not being able to do their work efficiently due to dependencies on the groups that were not present. Thus we were asked to create a discussion about this at the weekly meeting in the multi-project. It was decided to relax the office hour requirement, such that it is acceptable to not be physically present in one's group room --- instead each group has to be reachable by e-mail in the same time period. Groups are, however, encouraged to be physically present.
\end{description}

\section{Multi-Project Sprint Review}
In the end of this sprint, a final common meeting among all groups was held. The purpose of this meeting was to evaluate and reflect upon the decisions in this sprint and to evaluate the process.

It was suggested to add one or two days between the \gui sprint planning and the \db and \bd sprint plannings. Since the \gui groups have to process user story prioritization done by the external customers as well as incorporate new requirements into the product backlog, they need time to do so, before they can elicit derived requirements to the \db and \bd groups\kimnote{This is the first time you mention this, I would expect it to be mentioned in swdev\_method.tex}.

A grace period after the sprint planning meetings was also suggested, to allow user stories to be incorporated a few days into the sprint. We understand this is not strictly Scrum, however it is necessary because, for example, a \db user story might require a new user story in \gui or \bd to be selected. If sprints were shorter, the grace period may not have been as necessary as it is now, since the user story could be selected shortly after. But when having one month sprints much time will be wasted just because the user story could not be selected.

Furthermore, it was suggested that the three product backlogs could be merged into one document. This makes it easier to access the product backlog and also hopefully decreases overlap.

At last, the server was rebooted the day before the \gui sprint end meeting. Unfortunately, the server did not boot correctly, and the server was not working. This posed a problem, since it made it difficult to show the external customers the product. It was suggested that the server should not reboot just up to a sprint end.

We will take the appropriate actions to incorporate these suggestions into the next sprint.