\chapter{Sprint Review}\label{chap:sprint1_end}
This sprint had a number of tasks that we wanted to solve and in addition we introduced some changes to the development method. This chapter evaluates the tasks we planned and the development method followed during this sprint.

\begin{chapterorganization}
  \item in \sectionref{sec:s1_goals} we evaluate our goals for this sprint;
  \item in \sectionref{sec:s1_processeval} we evaluate the development method in this sprint;
  \item in \sectionref{sec:s1_multiprj_review} we describe the results of the multi-project sprint review meeting.
\end{chapterorganization}

\section{Sprint Goals Evaluation}\label{sec:s1_goals}
For this sprint, our main task was to setup continuous integration. The following evaluates the tasks described in \sectionref{sec:group_sprint_planning}.

\begin{description}
  \item[Build] We setup Jenkins to automatically build a project whenever a push is made to the Git repository of that project. The build times, however, are very slow. It can take more than the 10 minutes recommended by \textcite{beck2004} to build some jobs on Jenkins. In addition if there is a queue of multiple jobs it can take even longer. We add a user story to the product backlog to improve build times.

  For integrating code with the mainline, we use a simple merge strategy which allows developers to commit directly to the master branch. The builds have generally been unstable on Jenkins. This is a potential problem because it can block developers from writing code in apps that do not work. However, because there have not been any complaints from the developers, we assume that it is not a big problem in practice. We will not work further with branching strategies.
  \item[Documentation] Documentation is automatically generated each night from the entire code base and uploaded to a publicly available website. This works as required by the task.
  \item[Concolic Testing] We chose not to implement concolic testing due to it being too complex to implement compared to the benefit it would give us. It may be worth looking more into this in a future sprint if the developers express an interest in this.
  \item[Lint] Lint warnings and errors are automatically generated on Jenkins. We decided not to make a lint error break the build as to not punish groups for lint errors caused by code from previous years. Lint errors are seldom critical. By displaying the lint errors on the Jenkins page some groups have indeed worked on reducing the number of lint errors. We will therefore not implement build failure for lint errors.
  \item[Unit Testing] We setup unit testing such that they run as part of a build on Jenkins. When a test fails, the build fails. However, testing takes a long time because they run on an emulator which must start up as part of the test process.
  \item[UI Testing] We wanted to implement monkey testing, but because the plugin on Jenkins did not support multiple app monkey testing and unavailable APKs on the server, we did not manage to implement monkey testing on Jenkins. However, we have made an improvement to the Jenkins plugin which makes it easier to eventually implement monkey testing in a future sprint.
\end{description}

\section{Process Evaluation During Sprint}\label{sec:s1_processeval}
During sprint 1 we changed a few things in the development method of the multi-project:

\begin{description}
  \item[Customer Relations Role] The \emph{customer relations} role was initially vague in its description. Some understood it as being a product owner (PO) role, while others understood it only as a contact person for the external customers. We solved this by letting this role be the contact person for the external customers and assigning the PO role to other groups as explained below.
  \item[Multiple Product Owners and Meetings] Originally, a single sprint review meeting with the external customers was planned. Due to concerns from the semester coordinator, it was changed into several meetings. Last year, there was one sprint review meeting, and it was observed that the external customers were not interested in most of the progress the \db and \bd groups had made. Therefore we decided that the external customers should only attend a meeting where the \gui groups would present their progress as well as the relevant progress from \db and \bd (if any). 
  
  We seek a way of organizing the different teams in a structure that allows them to work with user stories relevant to their subprojects. \textcite{bird_davies_2007} describes four methods of organizing teams in Scrum of Scrums:
  \begin{enumerate}
    \item Independent groups with a customer and PO of their own. They also have their own backlog.
    \item Groups have a single common backlog and share the customer and PO
    \item Groups have their own customer and PO, and a backlog that takes items from a large common backlog
    \item Subproject has their own PO, customer, and backlog, and groups can serve as customers for other groups
  \end{enumerate}
  The first and second options are the simplest, but they can prove difficult in the multi-project, as groups from each subproject may be dependent on the other groups. For example, the \gui subproject may require changes to the database structure, which should be resolved by the \db subproject. The third option adds an extra layer on top of the previous described methods, which allows user stories from the external customer to be distributed among the individual groups. However, it still has the issue that the primary customer is the external customer. The final option, on the other hand, allows for example the \gui groups to be customers of the \bd groups. Thus, the external customer can be abstracted from internal needs and technicalities. This structure fits the needs for the multi project well. Specifically, there are three POs in the multi-project, one for each subproject. The customer for the GUI PO is the external customer, whereas the customers for the \db and \bd POs are the groups from the other subprojects. 
  
  We chose to present the fourth method to the groups of the multi-project who approved it. The PO roles have been designated to a group (and not a single person) in each subproject. This differ from the regular Scrum method, where a product owner is a single person.
  
  The result of this is the structure described in \sectionref{sec:project_overview}, where there are three POs, three sprint review meetings, and three sprint planning meetings.
  \item[Office Hours] In the beginning of the multi-project, it was decided that all groups must be present in their group rooms from 09:00--15:00. However, after the decision had been made, some groups indicated that they did not intend to follow this requirement. This meant that other groups complained about not being able to do their work efficiently due to dependencies on the groups that were not present. Thus we were asked to create a discussion about this at the weekly meeting in the multi-project. It was decided to relax the office hour requirement, such that it is acceptable to not be physically present in one's group room --- instead each group has to be reachable by e-mail in the same time period. Groups are, however, encouraged to be physically present. This way all groups agree on the office hours.
\end{description}

\section{Multi-Project Sprint Review}\label{sec:s1_multiprj_review}
In the end of this sprint, a final common meeting among all groups was held. The purpose of this meeting was to evaluate and reflect upon the decisions in this sprint and to evaluate the process.

It was suggested to add one or two days between the \gui sprint planning and the \db and \bd sprint plannings. Since the \gui groups have to process user story prioritization done by the external customers as well as incorporate new requirements into the product backlog, they need time to do so, before they can elicit derived requirements to the \db and \bd groups.

A grace period after the sprint planning meetings was also suggested, to allow user stories to be incorporated a few days into the sprint. We understand this is not strictly Scrum, however it is necessary because, for example, a \db user story might require a new user story in \gui or \bd to be selected. If sprints were shorter, the grace period may not have been as necessary as it is now, since the user story could be selected shortly after. But when having one month sprints much time will be wasted just because the user story could not be selected.

Furthermore, it was suggested that the three product backlogs are merged into one document. This makes it easier to access the product backlog and also hopefully decreases overlap.

At last, the server was rebooted the day before the \gui sprint end meeting. Unfortunately, the server did not boot correctly, and the server was not working. This posed a problem, since it made it difficult to show the external customers the product. It was suggested that the server should not reboot just up to a sprint end.
