\chapter{Sprint Review}\label{chap:sprint1_end}
This sprint had a number of goals we wanted to achieve. During the sprint there were some development method changes that we investigated.

At the end of sprint 0 a several meetings were held, as described in \sectionref{sec:meetings}. This chapter described the meetings where we had a significant role.

Finally a review of the sprint was made together with all the groups of the multi-project.

\section{Sprint Goals}
\todo{What did we manage to make. Compare with our release backlog as mentioned in the sprint planning chapter}

\section{Process Evaluation During Sprint}
During sprint 0 two changes to the development method of the multi-project were made:

\begin{description}
  \item[Multiple Product Owners and meetings] When the semester started the responsibilities described in \sectionref{sec:multi_project_group_roles} were assigned to groups. The role \emph{customer relations} was initially vague in its description. Some understood it as being a product owner (PO) role, while others understood it only as a contact person for the external customers. As the sprint end came up, we investigated how to do the sprint end. In connection with this the semester coordinator asked specifically how the meeting with the external customer will progress --- whether or not all groups should attend. To avoid a long and overcrowded meeting with the external customers, we investigated a way to make it so not all groups had to attend. We ended up with the structure described in \sectionref{sec:project_overview}, where there are three POs, and three sprint end and sprint planning meetings. The internal meetings did not exist originally, and enables the multi-project to deal with things not directly related to the external customers. With the addition of three actual POs, the customer relations role was also specified to include only contact with the external customers.
  \item[Office Hours] Initially when the semester started, it was decided that all groups must be in their group rooms from 09:00--15:00. However, some groups could not follow this requirement. This meant that some groups complained that they could not do their own work efficiently due to dependencies from the groups that were not present. Thus we were asked to take this up at the weekly meeting with the multi-project. There it was decided to relax the office hour requirement, such that it is acceptable to not be physically present in one's group room --- instead each group has to be contactable on email in the same time period. Groups are, however, encouraged to be physically present.
\end{description}

\section{Sprint End Meetings}
At the end of sprint 0 there were three sprint end meetings, one for each subproject. We attended as pigs the \gui sprint end, where the external customers were present. At the \db sprint end we attended as customers for the \db subproject. Finally we presented our work to the other groups in the multi-project at the \bd sprint end meeting.

Here we describe the \gui and \bd sprint end meetings. We do not describe the \db sprint planning meeting, as nothing of interest for us occurred at this meeting --- it was mostly relevant for the \gui groups.

%\begin{dates}
%  \item[17-03-2013] \gui sprint end
%  \item[18-03-2013] \db sprint end
%  \item[19-03-2013] \bd sprint end
%\end{dates}

\subsection{GUI Sprint End}
At the \gui sprint end three of the external customers attended. The \gui groups presented their work to these customers, while they provided feedback. The \gui POs noted this feedback. We attended as pigs, but one member of our group presented the meeting initially.

\subsection{Build \& Deployment Sprint End}
The \bd groups presented their work at the \bd sprint end. Our group presented the work we had done on continuous integration. We made a top-down presentation of Jenkins, showing the Jenkins front page and what Jenkins is capable of. In addition we showed the website for the central documentation. We did spend too much time on presenting our work, mostly because the person from our group who presented also moderated the meeting. Thus in the future a member of the group who does not moderate the meeting will present our work.

\section{Multi-Project Sprint Review}
In the end of this sprint, a final common meeting among all groups was held. The purpose of this meeting was to evaluate and reflect upon the decisions in this sprint and to evaluate the process.

It was suggested to add one or two days between the \gui sprint planning and the \db and \bd sprint plannings. Since the \gui groups have to process user story prioritization done by the external customers as well as incorporate new requirements into the product backlog, they need time to do so, before they can elicit derived requirements to the \db and \bd groups. 

A grace period after the sprint planning meetings was also suggested, to allow user stories to be incorporated a few days into the sprint. We understand this is not strictly Scrum, however it is necessary because, for example, a \db user story might require a new user story in \gui or \bd to be selected. If sprints were shorter, the grace period may not have been as necessary as it is now, since the user story could be selected shortly after. But when having one month sprints much time will be wasted just because the user story could not be selected.

Furthermore, it was suggested that the three product backlogs could be merged into one document. This makes it easier to access the product backlog and also hopefully decreases overlap.

At last, the server was rebooted the day before the \gui sprint end meeting. Unfortunately, the server did not boot correctly, and the server was not working. This posed a problem, since it made it difficult to show the external customers the product. It was suggested that the server was not interfered with in the time before sprint ends.

We will take the appropriate actions to incorporate these suggestions into the next sprint.