\chapter{Sprint Review}\label{chap:sprint1_end}
This sprint had a number of goals we wanted to achieve. During the sprint there were some development method changes that we investigated.

At the end of sprint 0 a several meetings were held, as described in \sectionref{sec:meetings}. This chapter described the meetings where we had a significant role.

Finally a review of the sprint was made together with all the groups of the multi-project.

\section{Sprint Goals}
\todo{What did we manage to make. Compare with our release backlog as mentioned in the sprint planning chapter}

\section{Process Evaluation During Sprint}
During sprint 0 two changes to the development method of the multi-project were made:

\begin{description}
  \item[Multiple Product Owners and meetings] When the semester started the responsibilities described in \sectionref{sec:multi_project_group_roles} were assigned to groups. The role \emph{customer relations} was initially vague in its description. Some understood it as being a product owner (PO) role, while others understood it only as a contact person for the external customers. As the sprint end came up, we investigated how to do the sprint end. In connection with this the semester coordinator asked specifically how the meeting with the external customer will progress --- whether or not all groups should attend. To avoid a long and overcrowded meeting with the external customers, we investigated a way to make it so not all groups had to attend. We ended up with the structure described in \sectionref{sec:project_overview}, where there are three POs, and three sprint end and sprint planning meetings. The internal meetings did not exist originally, and enables the multi-project to deal with things not directly related to the external customers. With the addition of three actual POs, the customer relations role was also specified to include only contact with the external customers.
  \item[Office Hours] Initially when the semester started, it was decided that all groups must be in their group rooms from 09:00--15:00. However, some groups could not follow this requirement. This meant that some groups complained that they could not do their own work efficiently due to dependencies from the groups that were not present. Thus we were asked to take this up at the weekly meeting with the multi-project. There it was decided to relax the office hour requirement, such that it is acceptable to not be physically present in one's group room --- instead each group has to be contactable on email in the same time period. Groups are, however, encouraged to be physically present.
\end{description}

\section{Sprint Meetings}
\todo{Major sprint meetings of the sprint where we were involved.}

\subsection{GUI Sprint End}
\todo{What happened here?}
\todo{Shall we briefly mention that we did some (although minor) work at the gui sprint planning}

\subsection{\bd Sprint End}
\todo{What happened here? In particular what did our group show?}

\subsection{\db Sprint Planning}
\todo{We were headhunted to assist in this meeting, due to difficulties.}

\section{Multi-Project Sprint Review}
\url{http://kortlink.dk/gb8f}
\todo{Girafmøde 17--03}
\todo{Mere tid mellem GUI sprint planning og de resterende sprint planning for at tillade GUI grupper mere tid til at sætte sig ind i deres release backlog}