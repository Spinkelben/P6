\chapter{Configuration Management}\label{chap:config_management}
Configuration management is an important if not crucial part of any large software project. Our group has taken the responsibility to setup and maintain this area. Configuration management can greatly ease the workflow of the developers and increase their productivity. This chapter describes how we plan and execute configuration management. We aim to make it as easy as possible for the developers, and therefore we will automate as much as possible.

\begin{chapterorganization}
  \item in \sectionref{sec:jenkins} we describe the continuous integration tool, Jenkins, that we use in the project;
  \item in \sectionref{sec:branching_strategy} we discuss branching strategies and adopt one of them;
  \item in \sectionref{sec:build_automation} we explain how we set up automation of the builds;
  \item in \sectionref{sec:test_automation} we investigate and implement automated testing;
  \item in \sectionref{sec:automated_lint} we investigate and implement automated linting;
  \item in \sectionref{sec:automated_documentation_gen} we explain how we set up automated documentation generation.
\end{chapterorganization}

\section{Jenkins}\label{sec:jenkins}
\todo{Rename evt. til continuous integration og forklar hvorfor vi vil have noget af det. Så kan vi skrive, at Jenkins er et værktøj til at udføre det. På den måde argumenterer vi hvorfor vi bruger Jenkins.}
Jenkins had been installed and setup by previous years. Jenkins is an open source tool for continuous integration \parencite{JenkinsWebsite}. It supports source control management tools such as Git, as well as build automation tools such as Maven. It is extensible via numerous available plugins. Jenkins allows for a sophisticated continuous integration setup, however the setup by previous years is rather basic and we want to improve it in several ways.

\subsection{Upgrading Jenkins and Plugins}
We inherited the old installation which had not been updated in a long time. Jenkins itself and all the Jenkins plugins had updates available. We updated everything to the newest versions.

There are a number of unused plugins enabled in Jenkins. We tried to disable these, but this resulted in failure to start Jenkins. We undid the changes and have decided not to remove these unused plugins as it will not hinder further configuration of Jenkins.

\subsection{Setting up Roles in Jenkins}
The inherited Jenkins installation was open to anybody. We modify the settings to sensible values. Our group has full configuration access, while everybody else has read access.

\section{Merging With Master}\label{sec:branching_strategy}
% PendingHead: http://martinfowler.com/bliki/PendingHead.html
% Continuous Integration: http://www.martinfowler.com/articles/continuousIntegration.html
\todo{Pending Head/pre-tested commit vs alle committer til master}

\section{Build Automation}\label{sec:build_automation}
\todo{Vi har fikset det (noget med APK-filer) + skeduleret det nightly (nighly builds) som det aller første, da det overhovedet ikke var automatiseret til at starte med. Efterfølgende har vi så sat det op til at build ved hvert push til master.}

\section{Test Automation}\label{sec:test_automation}
\dummy~\dummy~\dummy~
% Skal være tosset hurtig
\subsection{Unit Testing}
\dummy~\dummy~\dummy~\dummy~\dummy~\dummy~
\subsection{Integration Testing}
\dummy~\dummy~\dummy

\dummy~\dummy~\dummy~
\subsection{UI Testing / Monkey Testing}
\dummy~\dummy~\dummy~\dummy~\dummy~\dummy~
\subsection{Concolic Testing}
At the sprint planning a desire to do automatic checking of null pointer exceptions was expressed. To this end we investigated the possibility of utilizing concolic testing. Concolic testing automatically tests all execution branches \parencite{concolic_testing_2015}. These tests can check for null pointer exceptions.

\todo{Vi har kigget på:}

\begin{itemize}
  \item LIME testbench \url{http://www.tcs.hut.fi/Software/lime/}.
  \item ACTEve \url{http://delivery.acm.org/10.1145/2400000/2393666/a59-anand.pdf?ip=130.225.198.195&id=2393666&acc=ACTIVE%20SERVICE&key=36332CD97FA87885%2E1DDFD8390336D738%2E4D4702B0C3E38B35%2E4D4702B0C3E38B35&CFID=627209490&CFTOKEN=76801213&__acm__=1424945812_ab641600a5058e5b199a617b521e4023}
\end{itemize}

\todo{Vi burde også kigge på:}

\begin{itemize}
  \item LCT \url{http://users.ics.aalto.fi/ktkahkon/KahLauSaaKauHelNie-BYTECODE2011.pdf}
  \item jCUTE \url{http://osl.cs.illinois.edu/software/jcute/}
  \item jFuzz \url{http://people.csail.mit.edu/akiezun/jfuzz/documentation.html}
  \item CATG \url{https://github.com/ksen007/janala2}
\end{itemize}

\todo{Andreas: Write about why we don't want this.}

\section{Automated Lint Check}\label{sec:automated_lint}
Lint checking is static code analysis and scans the source code for potential bugs and improvements. We investigate the possibility of automating lint checks on the source code, as we suspect this will uncover a wide range of improvements.

A tool for linting Android project source files exists \todo{Hvad hedder det?}. It checks for potential bugs and optimization improvements for correctness, security, performance, usability, accessibility, and internationalization \parencite{AndroidLint}.

There are some important considerations to consider before linting the source code automatically. The code base is inherited from earlier years and no lint checking to our knowledge was performed on this. Linting the code will produce a considerable amount of warnings. It is therefore not possible for us to let a build fail should it contain any warnings.

In this first sprint, we set up automating the lint check in Jenkins to be performed on all builds. The warnings are presented in the build overview screen, and we hope that having a low number of lint warnings will be incentive enough for each group to fix warnings. We consider creating an even stronger incentive if the number alone is not enough. An example of such a incentive is having weekly or per sprint ``competitions'' where the group lowering the number of serious lint warnings the most will receive a small prize and forever be on a hall of fame.

Over time, we hope that the presence of serious lint warnings can be a reason to fail a build. We may adopt this practice in a later sprint, and to help speed up this adoption we may set it up as follows. As there are a large number of lint warnings already in the code, we will select a baseline day. Groups will not be \emph{punished} (i.e.\ the break fails) for lint warnings that were present before the baseline day. Only newly introduced lint warnings will be considered and punished. Fixing old lint warnings from before the baseline day will of course count positively. %Implementing this requires subtracting the old lint warnings from the new lint warnings using a diff tool.

%\todo{Write about how to diff lint results with baseline.}

%\todo{Write about how to ignore the warnings found many times.}

\section{Automated Documentation Generation}\label{sec:automated_documentation_gen}
\todo{Mads: Fill this in.}\\
We have tested Javadoc. We will use Doxygen. How we set it up in Jenkins.
