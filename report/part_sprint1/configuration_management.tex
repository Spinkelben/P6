\chapter{Configuration Management}\label{chap:config_management}
Configuration management is an important if not crucial part of any large software project. Our group has taken the responsibility to setup and maintain this area. Configuration management can greatly ease the workflow of the developers and increase their productivity. This chapter describes how we plan and execute configuration management. We aim to make it as easy as possible for the developers, and therefore we will automate as much as possible.

\begin{chapterorganization}
  \item in \sectionref{sec:jenkins} we describe the continuous integration tool, Jenkins, that we use in the project;
  \item in \sectionref{sec:branching_strategy} we discuss branching strategies and adopt one of them;
  \item in \sectionref{sec:build_automation} we explain how we set up automation of the builds;
  \item in \sectionref{sec:test_automation} we investigate and implement automated testing;
  \item in \sectionref{sec:automated_lint} we investigate and implement automated linting;
  \item in \sectionref{sec:automated_documentation_gen} we explain how we set up automated documentation generation.
\end{chapterorganization}

\section{Jenkins}\label{sec:jenkins}
\todo{Rename evt. til continuous integration og forklar hvorfor vi vil have noget af det. Så kan vi skrive, at Jenkins er et værktøj til at udføre det. På den måde argumenterer vi hvorfor vi bruger Jenkins.}
Jenkins had been installed and setup by previous years. Jenkins is an open source tool for continuous integration \parencite{JenkinsWebsite}. It supports source control management tools such as Git, as well as build automation tools such as Maven. It is extensible via numerous available plugins. Jenkins allows for a sophisticated continuous integration setup, however the setup by previous years is rather basic and we want to improve it in several ways.

\subsection{Upgrading Jenkins and Plugins}
We inherited the old installation which had not been updated in a long time. Jenkins itself and all the Jenkins plugins had updates available. We updated everything to the newest versions.

There are a number of unused plugins enabled in Jenkins. We tried to disable these, but this resulted in failure to start Jenkins. We undid the changes and have decided not to remove these unused plugins as it will not hinder further configuration of Jenkins.

\subsection{Setting up Roles in Jenkins}
The inherited Jenkins installation was open to anybody. We modify the settings to sensible values. Our group has full configuration access, while everybody else has read access.

\section{Merging With Master}\label{sec:branching_strategy}
% PendingHead: http://martinfowler.com/bliki/PendingHead.html
% Continuous Integration: http://www.martinfowler.com/articles/continuousIntegration.html
\todo{Pending Head/pre-tested commit vs alle committer til master}

\section{Build Automation}\label{sec:build_automation}
\todo{Vi har fikset det (noget med APK-filer) + skeduleret det nightly (nighly builds) som det aller første, da det overhovedet ikke var automatiseret til at starte med. Efterfølgende har vi så sat det op til at build ved hvert push til master.}

\section{Test Automation}\label{sec:test_automation}
\dummy~\dummy~\dummy~
% Skal være tosset hurtig
\subsection{Unit Testing}
During the initial investigation of the inherited code base we found some existing unit tests in the \gproject{Oasis-lib} project. These tests utilize the Android unit testing framework included in the Android framework \parencite{AndroidUnit}. The unit tests are run on an emulated android environment, or on actual android devices, if any are connected to the computer. We therefore decide to postpone any further investigation into unit testing frameworks. Instead, we focus on getting the existing unit tests to run, both locally and in Jenkins. The tests were immediately runnable through Android Studio 0.4.6\todo{check this version number}. However, Android Studio does something behind the scenes when it runs the tests, and therefore we could not run them from the command line, which means that we cannot run them in Jenkins. However, the new Android studio, version 1.0.2, uses Gradle exclusively to run the tests. This means that when the projects got migrated to the new version, we became able to run the tests with Jenkins. 
The other projects did not have any tests, so we created test projects, with examples of tests to verify that Jenkins could run the tests, and to show the other groups how to create new tests.
\subsection{Integration Testing}
\dummy~\dummy~\dummy

\dummy~\dummy~\dummy~
\subsection{UI Testing / Monkey Testing}
\dummy~\dummy~\dummy~\dummy~\dummy~\dummy~
\subsection{Concolic Testing}
At the sprint planning a desire to do automatic checking of null pointer exceptions was expressed. To this end we investigated the possibility of utilizing concolic testing. Concolic testing automatically tests all execution branches \parencite{concolic_testing_2015}. These tests can check for null pointer exceptions.

\todo{Vi har kigget på:}

\begin{itemize}
  \item LIME testbench \url{http://www.tcs.hut.fi/Software/lime/}.
  \item ACTEve \href{http://delivery.acm.org/10.1145/2400000/2393666/a59-anand.pdf?ip=130.225.198.195&id=2393666&acc=ACTIVE\%20SERVICE&key=36332CD97FA87885\%2E1DDFD8390336D738\%2E4D4702B0C3E38B35\%2E4D4702B0C3E38B35&CFID=627209490&CFTOKEN=76801213&__acm__=1424945812_ab641600a5058e5b199a617b521e4023}{LINK}
\end{itemize}

\todo{Vi burde også kigge på:}

\begin{itemize}
  \item LCT \href{http://users.ics.aalto.fi/ktkahkon/KahLauSaaKauHelNie-BYTECODE2011.pdf}{LINK}
  \item jCUTE \url{http://osl.cs.illinois.edu/software/jcute/}
  \item jFuzz \url{http://people.csail.mit.edu/akiezun/jfuzz/documentation.html}
  \item CATG \url{https://github.com/ksen007/janala2}
\end{itemize}

\todo{Andreas: Write about why we don't want this.}

\section{Automated Lint Check}\label{sec:automated_lint}
Lint checking is static code analysis and scans the source code for potential bugs and improvements. We investigate the possibility of automating lint checks on the source code, as we suspect this will uncover a wide range of improvements.

A tool for linting Android project source files exists \todo{Hvad hedder det?}. It checks for potential bugs and optimization improvements for correctness, security, performance, usability, accessibility, and internationalization \parencite{AndroidLint}.

There are some important considerations to consider before linting the source code automatically. The code base is inherited from earlier years and no lint checking to our knowledge was performed on this. Linting the code will produce a considerable amount of warnings. It is therefore not possible for us to let a build fail should it contain any warnings.

In this first sprint, we set up automating the lint check in Jenkins to be performed on all builds. The warnings are presented in the build overview screen, and we hope that having a low number of lint warnings will be incentive enough for each group to fix warnings. We consider creating an even stronger incentive if the number alone is not enough. An example of such a incentive is having weekly or per sprint ``competitions'' where the group lowering the number of serious lint warnings the most will receive a small prize and forever be on a hall of fame.

Over time, we hope that the presence of serious lint warnings can be a reason to fail a build. We may adopt this practice in a later sprint, and to help speed up this adoption we may set it up as follows. As there are a large number of lint warnings already in the code, we will select a baseline day. Groups will not be \emph{punished} (i.e.\ the break fails) for lint warnings that were present before the baseline day. Only newly introduced lint warnings will be considered and punished. Fixing old lint warnings from before the baseline day will of course count positively. %Implementing this requires subtracting the old lint warnings from the new lint warnings using a diff tool.

%\todo{Write about how to diff lint results with baseline.}

%\todo{Write about how to ignore the warnings found many times.}

\section{Automated Documentation Generation}\label{sec:automated_documentation_gen}
When working with a large code base, people will most likely not have insight in all parts of the code. Some kind of documentation is helpful in order to know how to use different libraries. However, the developers of the multi project do not want more documentation than absolutely necessary, nor do they want to use much time writing it. An consequence of being agile is that the system architecture is likely to change over time --- so it is important that little effort is required for updating the documentation. Another requirement is that the developers want the documentation for all projects in one place. Based on these requirements, we choose to use a documentation generator which automatically can generate documentation from the code rather than writing such documentation manually.

We investigate the Javadoc and Doxygen tools. Both tools are cross-platform and can be integrated with Jenkins. Javadoc can generate documentation in HTML and Doxygen can generate both HTML and \LaTeX. We will generate HTML documentation which will be hosted on our server.

Javadoc is the official documentation generator for Java and generates documentation based on special formatted comments embedded in the source code. This format is integrated in Android Studio, making the documentation easily accessible where needed. In addition, parts of the existing code already contain Javadoc comments. A problem with Javadoc, however, is that it follows the packages and classes referred in a Java file. A requirement for this is that the source code must be a correctly structure Java project in order for the tool to find the different classes. This makes it complicated to comply with the requirement of having documentation of all projects in one place, because a combination of all projects may not form a valid Java project. The source code of all projects cannot simply be copied into one directory and fed to the Javadoc tool.

The Doxygen tool is a cross-language documentation generator that supports the same documentation syntax as Javadoc. Doxygen does not follow the class dependencies but simply parses the specified files. The HTML-documentation is very similar to that of Javadoc, and because it is more convenient to use, we have chosen to use this tool for documentation generation.
\todo{Jenkins, ftp etc.}
\todo{Sources}

\href{http://www.stack.nl/~dimitri/doxygen/}{Doxygen link}\\
\href{http://www.oracle.com/technetwork/java/javase/documentation/index-jsp-135444.html}{javadoc link}
