\chapter{SW Configuration Management}\label{chap:config_management}
Software configuration management is an important if not crucial part of any large software project. The large size of the multi-project requires some form of software configuration management to handle and track the changes in the software.

Configuration management can greatly ease the workflow of the developers and increase their productivity. This chapter describes how we plan and execute configuration management. We aim to make it as easy as possible for the developers, and therefore we will automate as much as possible.

\kimnote{When you use the term configuration management then be aware that it have a special meaning. There are multiple standards that define how proper configuration management is done and measured. This is a great opportunity to talk about official methods and how you utilize these in this project. :)}

\begin{chapterorganization}
  \item in \sectionref{sec:SCM_vision} we identify a plan for the software configuration management and identify specific areas we want to improve or implement in the software configuration management practice;
  \item in \sectionref{sec:jenkins} we describe the continuous integration tool, Jenkins, that we use in the project;
  \item in \sectionref{sec:build_automation} we explain how we set up automation of the builds and discuss branching strategies;
  \item in \sectionref{sec:test_automation} we investigate and implement automated testing;
  \item in \sectionref{sec:automated_lint} we investigate and implement automated linting;
  \item in \sectionref{sec:automated_documentation_gen} we explain how we set up automated documentation generation.
\end{chapterorganization}

\section{Software Configuration Management Plan}\label{sec:SCM_vision}
Software Engineering Body of Knowledge (SWEBOK \todo{insert citation}) identifies some key areas of software configuration management.

\subsection{Organizational Context}
The responsibility of software configuration management is assigned to this group. As such, we vision the process will as unintrusive for the developers as possible.
\dummy~\dummy

\subsection{Configuration Item Identification}
\dummy

\subsection{Vision for Software Configuration Management}
\fxfatal{Mon ikke det skal placeres i introduction i stedet, da det ikke kun er vision for sprint 1?}
\dummy
\begin{description}
  \item[Build Management] We want automatic builds, such that the developer does not have to worry about it.
  \item[Process Management] \todo{Ensuring adherence to the organization's development process.}
  \item[Environment Management] \todo{Managing the software and hardware that host the system.}
  \item[Configuration Auditing] \todo{Ensuring that configurations contain all their intended parts and are sound with respect to their specifying documents, including requirements, architectural specifications and user manuals.}
  \item[Release Management] easy installation to tablets? Forskellige versioner, forskellige pictogrammer.
  \item[Development Environment] easy installing of developing environment on developer computers.
\end{description}

We do not want to set up a change control board (who can approve or reject change requests) --- not in agile mindset.

\subsection{Tool Selection}
Previous years already had revision control, and we take over their revision setup with Git \parencite{gitwebsite}. And Jenkins (explained later). Discuss the ``legacy''-aspect (how will the new SCM be applied to the old projects). Scope (how will the new tools be deployed --- entire organization or only specific projects?). Ownership (who is responsible for the introduction of the new tools?). Future plans (will the tools evolve in the future? New uses in the future?). Integrations (how are the tools integrated into eachother?)

\section{Jenkins}\label{sec:jenkins}
\todo{Rename evt. til continuous integration og forklar hvorfor vi vil have noget af det. Så kan vi skrive, at Jenkins er et værktøj til at udføre det. På den måde argumenterer vi hvorfor vi bruger Jenkins.}
Jenkins had been installed and setup by previous years. Jenkins is an open source tool for continuous integration \parencite{JenkinsWebsite}. It supports source control management tools such as Git \parencite{gitwebsite}, as well as build automation tools such as Maven \parencite{mavenwebsite}. It is extensible via numerous available plugins. Jenkins allows for a sophisticated continuous integration setup, however the setup by previous years is rather basic and we want to improve it in several ways.

\subsection{Upgrading Jenkins and Plugins}
We inherited the old installation which had not been updated in a long time. Jenkins itself and all the Jenkins plugins had updates available. We updated everything to the newest versions.

There are a number of unused plugins enabled in Jenkins. We tried to disable these, but this resulted in failure to start Jenkins. We undid the changes and have decided not to remove these unused plugins as it will not hinder further configuration of Jenkins.

\subsection{Setting up Roles in Jenkins}
The inherited Jenkins installation was open to anybody. We do not find this sensible as we need to control the build process. Allowing everybody access will likely end in someone modifying a setting without our knowledge. We intend to set up a mechanism for automatic build, therefore other people do not need the option to start builds. It might even interfere with the automatic build, if other people have access to Jenkins' configuration and build. 
\href{http://www.martinfowler.com/articles/continuousIntegration.html\#EveryoneCanSeeWhatsHappening}{Link to Martin Fowler}. We modify the access settings to allow us to have full configuration access while everybody else has read access. 

\section{Build Automation}\label{sec:build_automation}
The inherited project has no build automation, we decide to set it up in two stages. In the first stage we schedule all jobs to run nightly. This gives us some level of build automation and gives us time to investigate merge strategies and set it up properly. The nightly job is set to run every day at midnight.

The output of a build is an APK file, which is a package format used to distribute and install software onto Android. These files need to be signed before they can be installed. As such, we add a post-build task to Jenkins which signs and verifies the APK files as well as moves it into a common APK folder for all projects.

\subsection{Merging With Master}\label{sec:branching_strategy}
As the name \emph{continuous integration} suggests, code should be integrated into the mainline (or \emph{master branch}) of the project frequently \parencite{fowlerCI}. According to Martin Fowler, frequent merges ensure that merges generally will be small and easy to perform \parencite{fowlerFeatureBranch}. The master branch must be stable and always in a release-ready state. If it breaks, it should be the team's first priority to fix it. A consequence of this is that the whole team is affected when a developer introduces an error, which has a negative influence on the overall productivity.

We assess two strategies to accommodate these consequences of continuous integration: the \emph{direct commit strategy}\todo{selfopfundet navn. Må vi gerne kalde det det?} and the \emph{pre-tested commit strategy}. In the direct commit strategy, every developer integrates its code directly to the master branch. It is the developer's own responsibility that the code works. This is the simplest, and one may say, the most agile way of integrating code with the master branch. An illustration of the direct commit strategy can be seen in \figureref{fig:commit_stratagy_a}. A developer creates a branch from the master branch and develops its code (commits \mono{B} and \mono{C}) on this before merging it directly into the master branch (commit \mono{D}).

An alternative to integrating code directly with the master branch is to use pre-tested commits \parencite{fowlerPendingHead}. A pre-tested commit uses a special branch which is an intermediate place for building and testing code before it is merged into the master branch. The code will only be merged into the master branch if it passes the tests. This ensures that the master branch will always work, but the merging workflow will be more complex as the developer must pull from one branch and push to another. The strategy is illustrated in \figureref{fig:commit_stratagy_b}. The developer creates a branch from the master branch and develops its code on here. When completed, the code is merged with the intermediate branch (commit \emph{\mono{I}}), which will build and test the code before eventually merging it with the master branch (commit \mono{D}).

For the first sprint, we choose to implement the direct commit strategy, primarily because of its simplicity. It is important to get continuous integration up running so the developers can start to develop code, and it would be too costly to spend time implementing a pre-tested commit strategy as this will block the progress of all other developers. We are unsure about how the developers handle the increased responsibility, so in the preparation of the second sprint, we will evaluate this strategy and consider whether we should implement the pre-tested commit strategy instead.\todo{Husk at evaluére dette}
\todo{Evt. noget om peer-reviews (gerrit, pull-requests)}

\begin{figure}
\begin{subfigure}[b]{\linewidth}
\centering
\begin{tikzpicture}
% MASTER
\node[text width=2cm] at (0,1) {\mono{master}};
\node(master_a) [draw, circle, minimum size=0.5cm] at (2,1) {\mono{A}};
\node(master_head) [draw, circle, minimum size=0.5cm] at (5,1) {\mono{D}};
\node(master_head_label) [] at (5,2) {\mono{HEAD}};
\draw[->, >=latex] (master_head) -- (master_a);
\draw[->, >=latex] (master_head_label) -- (master_head);

% LOCAL
\node[text width=2cm] at (0,0) {\mono{dev\_branch}};
\node(local_b) [draw, circle, minimum size=0.5cm] at (3,0) {\mono{B}};
\node(local_c) [draw, circle, minimum size=0.5cm] at (4,0) {\mono{C}};
\draw[->,>=latex] (local_b) -- (master_a);
\draw[->,>=latex] (master_head) -- (local_c);
\draw[->,>=latex] (local_c) -- (local_b);
\begin{pgfonlayer}{background}
  \filldraw [line width=4mm,join=round,black!10]
      (-1, 1.2)  rectangle (6,0.8)
      (-1, 0.2)  rectangle (6,-0.2);
\end{pgfonlayer}
\end{tikzpicture}
\caption{The direct commit strategy}\label{fig:commit_stratagy_a}
\end{subfigure}\\
\begin{subfigure}[b]{\linewidth}
\centering
\begin{tikzpicture}
% MASTER
\node[text width=2cm] at (0,2) {\mono{master}};
\node(master_a) [draw, circle, minimum size=0.7cm] at (2,2) {\mono{A}};
\node(master_head) [draw, circle, minimum size=0.7cm] at (6,2) {\mono{D}};
\node(master_head_label) [] at (6,3) {\mono{HEAD}};
\draw[->, >=latex] (master_head) -- (master_a);
\draw[->, >=latex] (master_head_label) -- (master_head);

% PENDING HEAD
\node[text width=2cm] at (0,1) {\mono{intermediate}};
\node(ph_head) [draw, circle, minimum size=0.7cm] at (5,1) {\emph{\mono{I}}};
\draw[->, >=latex] (master_head) -- (ph_head);

% LOCAL
\node[text width=2cm] at (0,0) {\mono{dev\_Branch}};
\node(local_b) [draw, circle, minimum size=0.7cm] at (3,0) {\mono{B}};
\node(local_c) [draw, circle, minimum size=0.7cm] at (4,0) {\mono{C}};
\draw[->,>=latex] (local_b) -- (master_a);
\draw[->,>=latex] (ph_head) -- (local_c);
\draw[->,>=latex] (local_c) -- (local_b);

\begin{pgfonlayer}{background}
  \filldraw [line width=4mm,join=round,black!10]
      (-1, 2.2)  rectangle (7,1.8)
      (-1, 0.2)  rectangle (7,-0.2);
  \filldraw [line width=4mm,join=round,black!10]
      (-1, 1.2)  rectangle (7,0.8);
\end{pgfonlayer}
\end{tikzpicture}
\caption{The pre-tested commit strategy}\label{fig:commit_stratagy_b}
\end{subfigure}
\caption{Different branching strategies. A circles represents a commit and an arrow represents a reference to a commit.}\label{fig:commit_stratagy}
\end{figure}

\todo{Write about how we implemented this. Collab with git group?}

\subsection{Continuous Test}
After deciding upon the merge strategies, we improve the nightly build and test automation. Instead of building and testing nightly, we automatically build and test projects whenever new code is pushed to the master branch. This gives the developer fast feedback on the integration of the code, and we are likely to discover an error in the code faster than before. The automatic build is triggered by a git-hook on each repository which is a way to execute a script when specific git events occur. The script is executed whenever something is pushed to the repository. It sends an HTTP request to the Jenkins server which triggers builds of all projects that depend on the repository. Now that the automatic test setup is created, we examine the different types of automatic tests we can use on Android.

Because one of the ideas behind automated build is to give the developer fast feedback on the state of the code, the build and test process should be fast. The Extreme Programming (XP) development method states 10 minutes as a guideline for how long time a build should take \parencite{beck2004}. While we think this is a reasonable time, not all parts of the system can be thoroughly tested in that time. In such cases, Martin Fowler suggests that the fastest and general tests should be whenever a commit is pushed to the master branch, and slower tests can be triggered for later execution \parencite{fowlerCI}. We call these kinds of tests \emph{delayed tests}.\todo{Brug dette term fremover}

\todo{Vi har gjort apk-naming bedre (så de ikke hedder workspace)}

\subsection{Unit Testing}
During the initial investigation of the inherited code base we found some existing unit tests in the \gproject{Oasis-lib} project. These tests utilize the Android unit testing framework \todo{Caps?} included in the Android framework \parencite{AndroidUnit}. The unit tests runs in an emulated Android environment, or on actual Android devices if any are connected to the computer. We therefore decide to postpone any further investigation into unit testing frameworks. Instead, we focus on getting the existing unit tests to run, both locally and in Jenkins. The tests were immediately runnable through Android Studio 0.4.6 \todo{check this version number}. However, Android Studio does something behind the scenes when it runs the tests, and therefore we could not run them from the command line, which means that we cannot run them in Jenkins. However, the most recent Android studio, version 1.0.2, uses Gradle exclusively to run the tests. This means that when the projects are migrated to the new version, we are able to run the tests with Jenkins.
The other projects did not have any tests, so we create test projects, with examples of tests to verify that Jenkins can run the tests, and to show the other groups how to create new tests.
Currently, Jenkins starts a new emulator before each project is built. This adds a significant delay to the build. The build time for the entire app has increased from \SIrange[range-units = single]{20}{90}{\minute}. This is unfortunate, but for now we won't spend more time improving on this. We may return to this in a later sprint. Unit tests are the simplest tests we have, and we do not want these to be handled as delayed tests.

\subsection{Integration Testing}
\dummy~\dummy~\dummy
\todo{Måske noget om at komplicerede integration tests kan køres som delayed tests}
\dummy~\dummy~\dummy~
\subsection{UI Testing / Monkey Testing}
Writing UI tests can be labor intensive and the maintenance of the tests can get quite comprehensive. As an alternative, monkey testing can be used. Monkey testing is when a machine randomly presses buttons, acting like a monkey. That way the UI and the entire app is exercised. There is no guarantees that the full app will be tested, but there is almost no setup. The official Android SDK has monkey testing facilities build in. We will configure Jenkins to run monkey tests during the night, as delayed tests, when the server otherwise is idle. The other groups will be able to subscribe to tests reports of the individual objects \todo{Hvad er objects?}. That way they can get notified when issues are found during the monkey test.
\subsection{Concolic Testing}
Concolic testing is a way to statically analyze potential null pointer exceptions. Concolic testing automatically tests all execution branches in code \parencite{concolic_testing_2015}.

\todo{Vi har kigget på:}



\begin{itemize}
  \item LIME testbench \url{http://www.tcs.hut.fi/Software/lime/}.
  \item ACTEve \href{http://delivery.acm.org/10.1145/2400000/2393666/a59-anand.pdf?ip=130.225.198.195&id=2393666&acc=ACTIVE\%20SERVICE&key=36332CD97FA87885\%2E1DDFD8390336D738\%2E4D4702B0C3E38B35\%2E4D4702B0C3E38B35&CFID=627209490&CFTOKEN=76801213&__acm__=1424945812_ab641600a5058e5b199a617b521e4023}{LINK}
\end{itemize}

\todo{Vi burde også kigge på:}

\begin{itemize}
  \item LCT \href{http://users.ics.aalto.fi/ktkahkon/KahLauSaaKauHelNie-BYTECODE2011.pdf}{LINK}
  \item jCUTE \url{http://osl.cs.illinois.edu/software/jcute/}
  \item jFuzz \url{http://people.csail.mit.edu/akiezun/jfuzz/documentation.html}
  \item CATG \url{https://github.com/ksen007/janala2}
\end{itemize}

We investigate several frameworks for concolic testing, including LIME Testbench \parencite{conc-lime} and ACTEve \todo{Insert refs}. These frameworks are meant to be used with pure Java code \todo{also ACTEve??}, therefore we need to adapt our Android project code to support the frameworks. We do not think this work is worth the payoff, so we will not spend more time investigating this.

\section{Automated Lint Check}\label{sec:automated_lint}
Lint checking is static code analysis that scans the source code for potential bugs and improvements. We investigate the possibility of automating lint checks on the source code, as we suspect this will uncover a wide range of improvements.

An official tool, Android Lint \parencite{AndroidLint}, for linting Android project source files exists. It checks for potential bugs and optimization improvements for correctness, security, performance, usability, accessibility, and internationalization.

There are some important considerations to consider before linting the source code automatically. The code base is inherited from earlier years and no lint checking to our knowledge was performed on this. Linting the code will produce a considerable amount of warnings. It is therefore not possible for us to let a build fail if it contain any warnings.

In this first sprint, we set up automating the lint check in Jenkins to be performed on all builds. The warnings are presented in the build overview screen, and we hope that having a low number of lint warnings will be incentive enough for each group to fix warnings. We consider creating an even stronger incentive if the number turns out to be too small an incentive. An example of such an incentive is having weekly or per sprint ``competitions'' where the group lowering the number of serious lint warnings the most will receive a small prize and forever be on a hall of fame.

Over time, we hope that the presence of serious lint warnings can be a reason to fail a build. We may adopt this practice in a later sprint, and to help speed up this adoption we may set it up as follows. As there are a large number of lint warnings already in the code, we will select a baseline day. Groups will not be \emph{punished} (i.e.\ the break fails) for lint warnings that were present before the baseline day. Only newly introduced lint warnings will be considered and punished. Fixing old lint warnings from before the baseline day will of course count positively. %Implementing this requires subtracting the old lint warnings from the new lint warnings using a diff tool.

%\todo{Write about how to diff lint results with baseline.}

%\todo{Write about how to ignore the warnings found many times.}

\section{Automated Documentation Generation}\label{sec:automated_documentation_gen}
When working with a large code base, people will most likely not have insight in all parts of the code. Some kind of documentation is helpful in order to know how to use libraries. However, the developers of the multi-project do not want more documentation than absolutely necessary, nor do they want to use much time writing it. An consequence of being agile is that the system architecture is likely to change over time --- so it is important that little effort is required for updating the documentation. Another requirement is that the developers want the documentation for all projects in one place. Based on these requirements, we choose to use a documentation generator which automatically generates documentation from the code rather than writing such documentation manually.

We investigate the Javadoc \parencite{javadoc} and Doxygen \parencite{doxygen} tools. Both tools are cross-platform and can be integrated with Jenkins. Javadoc can generate documentation in HTML and Doxygen can generate both HTML and \LaTeX. We will generate HTML documentation which will be hosted on our server.

Javadoc is the official documentation generator for Java and generates documentation based on specially formatted comments embedded in the source code. This format is integrated in Android Studio, making the documentation easily accessible where needed. In addition, parts of the existing code already contain Javadoc comments. A problem with Javadoc, however, is that it follows the packages and classes referred in a Java file. A requirement for this is that the source code must be a correctly structured Java project in order for the tool to find the different classes. This makes it complicated to comply with the requirement of having documentation of all projects in one place, because a combination of all projects may not form a valid Java project. The source code of all projects cannot simply be copied into one directory and fed to the Javadoc tool.

The Doxygen tool is a cross-language documentation generator that supports the same documentation syntax as Javadoc. Doxygen does not follow the class dependencies but simply parses the specified files. The HTML-documentation is very similar to that of Javadoc, and because it is more convenient to use, we have chosen to use this tool for documentation generation.

\subsection{Documentation Generation in Jenkins}
We have configured a Jenkins job to generate the documentation for all projects. This job pulls the most recent state of the master branch of each project and executes the Doxygen tool on all code. On the current code base, the doxygen tool uses 3--7 minutes to generate the documentation \todo{Why the large span?}. Because we do not want to block more important jobs, such as building and testing new commits, we choose to run this job nightly.
